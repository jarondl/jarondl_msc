\chapter{Completed Work}


This work was published at Physical Review E \cite{de_leeuw_diffusion_2012},
and reprinted here in \autoref{sec:papers}. In the
following sections we will provide an overview
of the main results.


%%%%%%%%%%%%%%   Overview over article should not include too much equations
%%  bottom line - take home message - the derivations should ref the article

%%  
%%  remove the techinalities - 
%%     using the procedure we have calculated for.. blah blah
%%       see figure. (leave plots only if neccessary to buttom line).
%%
%%  geometry plots - maybe try with normal scale.
%%                   if that doesn't work, leave the figure out.



%%%%%%%%%%%%%%%%%%%%%%%%%
\section{Regular diffusion in the $2d$ sparse network}

We studied models with a rate equation of the type $w_{ij}=w_0 \exp{-r_{ij}/\xi}$ 
(see section II of the attached article). In particular, we have mapped the
$d=1$ model to known results of the transport and survival \cite{alexander_excitation_1981},
and compared this with our numerics. 


The $d=2$ case was studied using RG techniques in \cite{amir_mean-field_2008,*amir_localization_2010}, 
where it was suggested that anomalous diffusion takes place for sparse configurations.
However, our analysis indicates that for any amount of sparsity,
the diffusion is regular ($S(t)\sim (2d)(Dt)$).

%%%%%%%%%%%%%%%%%%%%%%%%%
\section{The effective range hopping procedure}


Feeling unsatisfied with the linear expression for the diffusion
in sparse systems,
we set off for a better diffusion approximation.

The basic idea behind \emph{ERH} (effective range hopping) is that in the linear
expression, nearby sites 
with $r\ll 1$ (and therefore $w \gg 1$ ) are over represented in
the diffusion coefficient calculation. While the transition to
these sites is indeed high, the distance covered is not enough
to form a percolating cluster. Therefore, we use a threshold based
on percolation theory and flat-down the rates higher then this threshold.


Using the \emph{ERH} procedure we have calculated the diffusion for several models.


For the $d=2$ ordered lattice with nearest neighbor random hopping:
%
\begin{align}
D_{\tbox{ERH}} =  \left[\frac{1}{2}w_c + \frac{1}{2}\int_0^{w_c} w\tilde{f}(w) dw\right]r_0^2
\end{align}
%
In absence of the disorder (i.e.\ all the rates are equal),
the known result $D = w_0r_0^2$ is restored.


For the degenerate hopping model, 

\begin{align}
D_{\tbox{ERH}} &=  \mathrm{EXP}_{d{+}2}\left(\frac{1}{s_c}\right)  \  \eexp{-1/s_c}  \ D_{\tbox{linear}}
\end{align}
%


For the Mott Hopping model,
%
\begin{align}
D_{\tbox{ERH}} &=\mathrm{EXP}_{d{+}3}\left(\epsilon_c\right)  \  \eexp{-\epsilon_c}  \ D_{\tbox{linear}}
\end{align}
%


For the flat-profile banded $1d$ model,
\begin{align}
D_{\text{ERH}} \ \ = \ \ 
\ \frac{1}{\sigma}\left[ 
\left(1+\frac{n_c}{2b}\sigma\right)\eexp{-\frac{n_c}{2b}\sigma} - \eexp{-2\sigma}
\right] \ \tilde{b} w_0
\end{align}

%%%%%%%%%%%%%%%%%%%%%%%%%%%%%%%%%%%%%%%%%%%%%%%%%%%%
%%%%%%%%%%%%%%%%%%%%%%%%%%%%%%%%%%%%%%%%%%%%%%%%%%%%
\chapter{Preliminary analysis}




%%%%%%%%   I need to add dts.tex here.
%%%%%%%%   I could also add D(g_s) and say that it failed

%%%%%%%%%%%%%%%%%%%%%%%%%%%%%%%%%%%%%%%%%%%%%%%%%%%
\section{The quasi one dimensional rate equation}

The quasi one dimensional system, is a one dimensional system
with a bandwidth greater than $1$ (going beyond nearest neighbors). 
We continue our theme by studying the survival and transport properties 
in those systems, specifically in sparse configurations.


In recent studies \cite{bodyfelt_unpub} an interesting structure of
PN (participation number) was observed in a similar model. In this model,
$N$ equal masses are connected by springs up to a bandwidth $b$. In addition to these
springs, there are "pinning" springs from each mass to its initial position, with
strength $\epsilon_n$, as described by the classical Hamiltonian:
\begin{align}
\mathcal{H} &= \sum_n \left( \frac{p_n^2}{2m} +\epsilon_n\frac{x_n^2}{2}+ \frac{1}{2}\sum_{j} \frac{1}{2}k_{nj} (x_n-x_j)^2 \right) \\
            &= \sum_{n} \frac{p_n^2}{2m} +  \frac{1}{2}\vec{x}^{\intercal} K \vec{x} \\
            K_{nm} &= \begin{cases} 
            \sum_{l\ne n} k_{ln} +\epsilon_n \quad &\textrm{if}\quad n=m \\ 
            - k_{nm}  \quad &\textrm{if}\quad n\ne m
            \end{cases}\\
            \vec{x} &= (x_1,x_2,\ldots,x_N)
\end{align}
The disorder is introduced by randomized $k_{ij}$ and $\epsilon_n$.

As can be seen
in \autoref{fig:PN_kottos}, for the low eigenvalues, instead of the 
expected large PN, we have some decrease. After that, follows a period
of large PN, followed by another segment with somewhat smaller PN.
Another quantity we calculated is the Thouless conductance (see \autoref{sec:anderson}), 
which we expect to correlate with $PN$. Disregarding major numerical issues,
the correlation seems to hold, see \autoref{fig:PN_g_scatter}.






\begin{figure}[H]
  %\subfloat[Low sparsity]{
  \begin{subfigure}{.45\linewidth}\centering
    \includegraphics[width=0.99\linewidth]{pta_nopin}
    \caption{PN vs $\lambda$}\label{fig:PN_kottos_nopin}
  \end{subfigure}
  %
  \begin{subfigure}{.45\linewidth}\centering
     \includegraphics[width=0.99\linewidth]{pta_ev_nopin}
  \caption{Counting function (CDF) $\mathcal{N}(\lambda)$}\label{fig:ev_dist}
  \end{subfigure}\\ % end of row
  %
 % \caption{}
 % \end{figure}
 % \begin{figure}[H]
  %
    \begin{subfigure}{.45\linewidth}\centering
    \includegraphics[width=0.99\linewidth]{pta_pin}
    \caption{PN vs $\lambda$, with diagonal disorder}\label{fig:PN_kottos_pinning}
  \end{subfigure}     
  %
  \begin{subfigure}{.45\linewidth}\centering
     \includegraphics[width=0.99\linewidth]{pta_ev_pin}
  \caption{Counting function (CDF) $\mathcal{N}(\lambda)$ with diagonal disorder}\label{fig:ev_dist_pin}
  \end{subfigure}
    \begin{subfigure}{.45\linewidth}\centering
    \includegraphics{pta_scatter_g}
  \caption{PN vs $g$}\label{fig:PN_g_scatter}
  \end{subfigure}
  \caption{For all the plots $N=1000$ and $b=5$. 
           In blue $\sigma=0.1$, green $\sigma=0.6$ and red $\sigma=10$, where higher
           $\sigma$ means more sparsity.
           %%%%
           In (\subref{fig:PN_kottos_nopin}) with low $\sigma$ (blue) we see two groups of PN values.
           This distinction is lost for higher $\sigma$ (red).
           %%
           In (\subref{fig:ev_dist}) we plot the cumulative distribution of
           the eigenvalues, presenting clear diffusive behavior, for any $\sigma$.
           %%%%%% pinning
           In (\subref{fig:PN_kottos_pinning}) and (\subref{fig:ev_dist})
           we see the effects of diagonal disorder (pinning)
           on the spectrum. We see clearly that the lowest eigenvalues 
           are affected, and their corresponding PN is much lower.
           %%%g 
           In (\subref{fig:PN_g_scatter}) we compare the Thouless conductance
           $g$ with the participation number. 
           The presented points do seem to be correlated, but due to numerical issues,
           many values of $g$ are below the precision limit (seen as a vertical line in the plot), 
           up to $936$ from $1000$ for $\sigma=10$. 
  }
\end{figure}



(\subref{fig:PN_kottos_nopin}\subref{fig:PN_kottos_sparse}\subref{fig:PN_kottos_pinning}) - {\bf PN} as a function of $\lambda$.
  For all the plots $N=1000$ and $b=5$. 
  The off diagonal elements are 
  $w_{ij} = \eexp{-\epsilon} \quad \, \epsilon \in \textrm{uniform} [0,2\sigma]$.
  For (\subref{fig:PN_kottos_pinning}\subref{fig:PN_kottos_sparse}), the diagonal elements are $w_{ii} = -\sum_j w_{ij}$, while in 
  the lower plot of (\subref{fig:PN_kottos_pinning}) we added a uniform 
  disordered parameter $x \quad \in \textrm{uniform} [0,0.3]$ to the diagonal.
  Comparing the two plots in \subref{fig:PN_kottos_pinning}, our initial numerical analysis 
  shows that the low-eigenvalue decrease
  is caused by the diagonal disorder. This disorder removes the one special 
  $\lambda=0$ eigenmode, and decreases the PN of several other modes.
  In \subref{fig:PN_kottos_sparse}, we see that once we use a much wider
distribution of off diagonal elements, i.e. increase the \emph{sparsity}, 
the two segment structure disappears.
  (d) - comparing the Thouless conductance $g$ with the participation number. 
  The presented points do seem to be correlated, but due to numerical issues,
  many values of $g$ are below the precision limit (seen as a vertical line in the plot), up to $936$ from $1000$ for $\sigma=10$. In our numeric 
  algorithm, the eigenvalue precision is relative to the highest eigenvalue. For wide distributions
  of eigenvalues, that means that the numerical precision of the low eigenvalues is poor,
  and the uncertainty is larger than the phase induced shift.





%%%%%%%%%%%%%%%%%%%%%%%%%%%%%
\section{The quasi one dimensional Hamiltonian}

In this section the numerical work was done by our collaborators,
Eli Halperin and Tsampikos Kottos of Wesleyan university.

In this model , described in \autoref{sec:quasi_bg}, initial analysis of
the numerical results reveals that the sparsity affects the diffusion coefficient.
We see that as the system becomes more sparse, the diffusion coefficient is suppressed. 
As might be expected, the lower bandwidth ensembles are more
susceptible, as the system is more vulnerable to disconnections.


\begin{figure}
\includegraphics{dts_D2}
\caption{The transient diffusion coefficient for quantum spreading in
a banded sparse network. The ratio $D/D_0$ between the numerical diffusion 
coefficient and the expected value decreases as $\sigma$ increases. The effect
is stronger for low bandwidth.}
\end{figure}


Our first question was whether our rate equation analysis might
hold also for the quantum case, because in first order perturbation theory,
the transition probability $p_{nm}$ is proportional to $\left|\bra n|\mathcal{H} | m \ket\right|^2$.
Naively using our previously obtained suppression factor $g_s = \frac{D_{ERH}}{D_{\textrm{linear}}}$,
did not prove to be sufficient, as can be seen in \autoref{fig:d_vs_g}.

\begin{figure}
\includegraphics{dts_D2_vs_gs_loglog}
\caption{Scaled $D$ vs $g_s$. }\label{fig:d_vs_g}
\end{figure}
