\chapter{Background}

 
% Also, some of the MSc_RP intro.

% Analogy to resistor network and to springs and masses.


\section{Transition matrices}

Our system is a $d$-dimensional network, and it is
described by its nodes and edges. The nodes represent 
either {\em sites} in real space or {\em levels} in energy space, 
and our particle (or particles, see \ref{sec:discreteness}) may
occupy them. The edges of the network represent transition rates 
between the nodes. The dynamics of the model are described by 
a rate equation:
%
\begin{align}
\frac{dp_n}{dt} \quad = \quad \sum_m w_{nm}p_m
\end{align}
%
Where the transition matrix 
%
\begin{align}
\mathbf{W} \quad =  \quad \left[ w_{nm}\right]_{N\times N}
\end{align}
%
Is the crux of this work.

Because of conservation of probability, we set the diagonal
elements to
%
\begin{align}
w_{nn}\quad = \quad -\sum_{m(\ne n)} w_{mn}
\end{align}
%


%%%%%%%%%%%%%%%%%%%%%%%%%%%%%%
\section{Definition of the model}


The proposal regards a spatial network in which each node
has a defined location $x_n$. The transition rates $w_{nm}$
are constructed by the expression
%
\beq
w_{nm} \quad = \quad w_0 \eexp{-\epsilon_{nm}}B(x_n-x_m)
\eeq
%
where $B(r)$ describes the dependence of the coupling on the
distance between sites, and $\epsilon$ represents activation
energies required for a transition.


The $B(r)$ matrix is a function of a euclidean distance matrix,
a class a matrices that is widely researched 
\cite{skipetrov_eigenvalue_2011, goetschy_non-hermitian_2011,mezard_spectra_1999, bogomolny_spectral_2003}
both in physics and in mathematics.


%%%%%%%%%%%%%%%%%%%%%%%%%%%%%%
\subsection{Applications of the model}

\rmrk{TODO: rewrite this}


The models that we address are related and motivated  
by various physical problems, for example: 
phonon propagation in disordered solids \cite{phn1,phn2,amir}; 
Mott hopping conductance \cite{mott,miller,AHL,Halp,pollak,VRHbook};
transport in oil reservoirs \cite{aa1,aa2};
conductance of ballistic rings \cite{kbd};
and energy absorption by trapped atoms \cite{kbw}. 
%
Optionally these models can be fabricated by combining oscillators: 
say mechanical springs or electrical RC elements. 


%%%%%%%%%%
\subsection{The degenerate random site hopping model} \label{sec:degenerate_random_hopping}

In this specific model, we set all of the activation energies ($\epsilon_{nm}$)
to be equal. We can choose $\epsilon=0$ without loss of generality. For $B(r)$, 
we have selected a simple exponential dependence with a parameter $\xi$.
Summing this up we write:
%
\begin{align}\label{eq:random_hopping_rates}
w_{nm} \quad = \quad w_0 B(x_n-x_m) \ = \ w_0 \eexp{-(x_n-x_m)/\xi}
\end{align}
%
Now we have defined the rates, but to characterize the system we also
have to define the location of the sites. In this model, $N$ sites are distributed
uniformly randomly on a periodic $d$-dimensional hypercube of volume $L^d$. The
local distribution of sites around any arbitrary point in this space is:
%
\begin{align}\label{eq:site_distribution}
\rho(r)dr \ =\ \frac{\Omega_d r^{d-1}}{r_0^d} dr
\end{align}
%
$\Omega_d$ and $r_0$ are defined in \autoref{sec:spacing}

%%%%%%%%%%%%
\subsection{The Mott hopping model}\label{sec:mott_hopping}

This model is an extension of the degenerate model to include
random values for $\epsilon$. $B(r)$ stays the same as in the degenerate model.
The rates are now:
%
\begin{align}\label{eq:mott_hopping_rates}
w_{nm} \quad = \quad \ w_0 \eexp{-\epsilon_{nm}}\eexp{-(x_n-x_m)/\xi}
\end{align}
%
And the combined distribution of $\epsilon$ and $r$ is now:
%
\begin{align}\label{eq:mott_distribution}
\rho(r)f(\epsilon)drd\epsilon \ &=\ \frac{\Omega_d r^{d-1}}{r_0^d} f(\epsilon) dr d\epsilon \\
f(\epsilon)\ &= 
  \begin{cases} 
    1 &\textrm{if   } \epsilon \in [0,\sigma] \\
    0 &\textrm{otherwise}
  \end{cases}
\end{align}
%


%%%%%%%%%%%%%
\subsection{The banded $1d$ lattice model}

In this model, we have a $1d$ lattice (i.e.\ equal spacing between the sites), 
with the rates :
%
\begin{align}
w_{nm} \quad = \quad \eexp{-\epsilon_{nm}}w_0 B(x_n-x_m)
\end{align}
%
The profile $B(r)$ allows connections to finite number of neighbors,
meaning that the resulting matrix is banded. This model might also
be called quasi-$1d$ because transitions could occur over more
than one path, removing the "bottle-neck" effect of $1d$ systems.


%%%%%%%%%%%%%%
\section{Sparsity}


\section{Transport and diffusion}


\section{Variable range hopping}


\section{Anderson localization}


%%%%%%%%%
\section{Analogy to an electrical circuit}

We may think of a "ladder" of capacitors and resistors as portrayed in the diagram.



%%%%%%%%%
\section{Analogy to a masses and springs mesh} 

The same network may describe equal masses (which 
we can simply define as $1$) distributed in space,
connected by springs. The locations of the masses will
obey a second degree differential equation:


%%%%%%%%%
\section{Spreading of wavepackets in the quantum case}





