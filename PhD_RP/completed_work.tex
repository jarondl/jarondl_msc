\chapter{Completed Work}

% Use drawings and slides from the theory lunch (actual and prelim talk)

% Also anything I have in the notes which is worthy.
% For example distance statistics. If I did not contribute, should be in the Appendix.
% Half cooked ideas..

%%%%%%%%%%%%%%%%%%%%%%%%%
\section{The $d=1$ random site model with nearest neighbor hopping}

In $d=1$, a network with randomly distributed sites is equivalent to
an ordered lattice with random transition rates. To make this mapping
we have to know the spacing distribution for this system. We find the
spacing distribution in \ref{sec:spacing}. A $1d$ lattice with
a known transition rates distribution is exactly solvable \cite{alexander}.
Basically, it is analogous to adding connectors in series, and 
the diffusion coefficient is the inverse of the combined resistivity
%
\begin{align}
D \ =\ \left(\frac{1}{N} \sum_n \frac{1}{w_{n,n-1}}\right)^{-1}
\end{align}
%
Plugging in the transition rate distribution from \ref{eq:rate_dist1d},
(with the same notation of $s \equiv \xi/r_0$ and converting the sum to an integral, we obtain
%
\begin{align}
D \ =\ \left(\frac{1}{N} \int_0^{w_0} \frac{s w^{s-1}dw}{w_0^s}\right)^{-1} \
\ =\ [s>1]\quad \frac{s-1}{s}w_0
\end{align}
% TODO : something is wrong with the N in this expression
This result is valid only for $s>1$. For $s<1$, $D=0$ and we have
subdiffusion, for which the survival probability and spreading 
are also exactly solvable \cite{alexander}, leading to:
%
\begin{align}
S(t)           \ &\sim \ t^{2s/(1+s)} \\
\mathcal{P}(t) \ &\sim \ t^{-s/(1+s)}
\end{align}
%
As explained in \ref{sec:spectrum}, the cumulative spectral distribution
can be deduced from $\mathcal{P}(t)$:
%
\begin{align}
\mathcal{N}(\lambda) \ \sim\ \lambda^{s/(1+s)}
\end{align}
%
Our numerics agree with these results, as seen in the left upper
panel of \ref{fig:spectral}.



%%%%%%%%%%%%%%%%%%%%%%%%%
\section{The ERH procedure}

The diffusion coefficient is defined by the dependence
of the variance in time:
%
\begin{align}
\textrm{Var}(n) \quad =\quad 2d\ D\ t
\end{align}
%
Within the transient diffusion stage,
assuming we start at a specific node $n$, the variance after time
$t$ will be
%
\begin{align}
\textrm{Var}(n) = \sum_m w_{nm} t (x_m-x_n)^2
\end{align}
Therefore, the diffusion coefficient for evolution 
starting at node $n$ is
%
\begin{align}
D_n \quad=\quad \frac{1}{2d}\sum_m (x_m-x_n)^2 w_{nm}
\end{align}
%
Averaging over the starting node, and converting the sum to an integral,
we have
%
\begin{align}\label{eq:linear_D}
D_{\tbox{linear}}  \ \ = \ \ \frac{1}{2d}\iint w(r,\epsilon) \ r^2  \ \rho(r,\epsilon) \ d\epsilon dr 
\end{align}
%

This expression describes the spreading in absence of disorder correctly
for arbitrary long time. However, in a sparse disordered system such as ours,
the possibility of transport is a bit less trivial. Therefore, we suggest
a method to approximate the diffusion, which takes percolation into account.


The basic idea behind \emph{ERH} is that in the linear
expression, nearby sites 
with $r\ll 1$ (and therefore $w \gg 1$ ) are over represented in
the diffusion coefficient calculation. While the transition to
these sites is indeed high, the distance covered is not enough
to form a percolating cluster. Therefore, we use a threshold based
on percolation theory and flat-down the rates higher then this threshold.

The rate threshold $w_c$ is determined using the following expression
%
\begin{align}\label{eq:threshold}
\iint_{w(r,\epsilon)>w_c} \rho(r,\epsilon)drd\epsilon \ \ = \ \ n_c
\end{align}
%
Where $n_c$ is the number of connected sites necessary to form a percolation
cluster (see \rmrk{TODO :ref to section explaining this better}).


Next, we suppress the rates higher than $w_c$ to have the value $w_c$,
and use the linear expression \ref{eq:linear_D}, to obtain:
%
\begin{align}\label{eq:ERH_D}
D_{\tbox{ERH}} = \frac{1}{2d}\iint \min\{w(r,\epsilon),w_c\} \ r^2  \ \rho(r,\epsilon) \ d\epsilon dr
\end{align}
%


%%%%%%%%%%%%%%%%%%%%%%%%%
\section{The ERH calculation for several types of network}
%%% This section is actually a collection of subsections.
%%%%%%%%%%%%%%%%%%%%%%%%%%
\subsection{The $d=2$ ordered lattice model, with n.n. hopping}

The model is a  two dimensional grid, with only nearest neighbor hopping possible. 
Each site has $4$ neighbors of equal distance, meaning that
the distribution of sites surrounding each site is simply
%
\begin{align}
\rho(r,\epsilon) \ = \ 4\ f(\epsilon)\delta(r-r_0) 
\end{align}
%
It is well known from percolation theory \cite{someonethatsaysthis} that in this model the number of necessary connections per site is $n_c=2$. We can
find $w_c$ from \ref{eq:threshold},an put this into the ERH 
expression \ref{eq:ERH_D} to have:
%
\begin{align}
D_{\tbox{ERH}} =  \left[\frac{1}{2}w_c + \frac{1}{2}\int_0^{w_c} w\tilde{f}(w) dw\right]r_0^2
\end{align}
%
Where we converted the $f(\epsilon)d\epsilon$ integral to a $\tilde{f}(w)dw$
integral. In absence of the disorder (i.e.\ all the rates are equal),
the known result $D = w_0r_0^2$ is restored.


%%%%%%%%%%%%%%%%%%%%%%%%%%
\subsection{The degenerate hopping model}

In this model the sites are randomly distributed in a $d$ dimensional 
space, all with $\epsilon=0$. Every site is connected to all of the other
sites, with the transition rates according to \ref{eq:random_hopping_rates}.

Since $w$ depends only on $r$, we may define the ERH threshold by
$r_c$ instead of $w_c$. Rephrasing equation \ref{eq:threshold}, and 
using $\rho(r,\epsilon)$ from \ref{eq:hopping_dist}, $r_c$ is determined according to
%
\begin{align}
\int_0^{r_c} \frac{\Omega_d r^{d-1} dr}{r_0^2} \ = \  n_c 
\end{align}
%
With the simple solution:
%
\begin{align}
r_c  &\equiv \left(\frac{d}{\Omega_c}n_c\right)^{1/d}  r_0 \\
w_c &= w_0\exp\left(-\frac{r_c}{\xi}\right) 
\end{align}
%

Now we shall put all of the above into the ERH expression \ref{eq:erh_D}.
The solution involves the incomplete gamma function \cite{gamma},
%
\begin{align}
\Gamma(\ell{+}1,x) = \int_0^x r^{\ell} \eexp{-r} dr = 
\ell! \ \mbox{EXP}_{\ell}(x)  \ \eexp{-x}
\end{align}
%
And the polynomial
%
\begin{align}
\mathrm{EXP}_{\ell}(x) \ \ = \ \ \sum_{k=0}^{\ell} \frac{1}{k!} \, x^k
\end{align}
% 
The integral is only over $dr$, and it is split  
into the domains ${0<r<r_c}$ and ${r>r_c}$. 
Namely, 
%
\begin{align}
D_{\tbox{ERH}} &= 
\frac{w_0\Omega_d}{2d}\int_0^{r_c} \eexp{-r_c / \xi} \frac{r^{d+1}}{r_0^d} dr 
\nonumber 
\quad+\quad \frac{w_0\Omega_d}{2d}\int_{r_c}^\infty \eexp{-r / \xi} \frac{r^{d+1}}{r_0^d} dr 
\nonumber \\
&= \frac{w_0\Omega_d}{2d} \eexp{-r_c / \xi} \frac{r_c^{d+2}}{d+2} \frac{1}{r_0^d}
\nonumber 
\quad+\quad \frac{w_0\Omega_d}{2d} \frac{\xi^{d+2}}{r_0^d} \Gamma\left(d+2,\frac{r_c}{\xi}\right)
\nonumber \\
&= \frac{w_0\Omega_d \xi^{d+2}}{2d(d+2)r_0^d}\Gamma\left(d+3,\frac{r_c}{\xi}\right)\\
&= \mathrm{EXP}_{d{+}2}\left(\frac{1}{s_c}\right)  \  \eexp{-1/s_c}  \ D_{\tbox{linear}}
\end{align}
%
Where $D_{\tbox{linear}}$ is formally $D_{\tbox{ERH}}$ with $r_c=0$, i.e.:
%
\begin{align}
D_{\tbox{linear}} \ \ = \ \  
\frac{(d{+}1)!\,\Omega_d}{2d} \, s^{d{+}2} \, w_0 r_0^2
\end{align}



%%%%%%%%%%%%%%%%%%%%%%%%%%%%
\subsection{The Mott hopping model}

%%%%%%%%%%%%%%%%%%%%%%%%%%%
\subsection{The Quasi one dimensional model}






