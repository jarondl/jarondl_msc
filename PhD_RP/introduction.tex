\chapter{Introduction}


This proposal is about simple dynamical networks, with dynamics
that are described either by a stochastic rate equation
%
\begin{align}
  \frac{dp_n}{dt} \quad = \quad \sum_m w_{nm}p_m
\end{align}
%
or by the Schr\"{o}dinger equation:
%
\begin{align}
  \frac{d\psi_n}{dt} \quad = \quad \sum_m \mathcal{H}_{nm}\psi_m
\end{align}
%

Regardless of the governing equation, the focus is
on disordered sparse networks, with wide distribution of energies. 
Two related aspects of these networks will be studied: their energy spectrum,
and their spreading properties. More specifically, we wish to determine
whether the networks are diffusive ($S(t)\sim t$), or sub-diffusive, and
if they are indeed diffusive, what is the diffusion coefficient. 




To simplify the discussion, we concentrate on $d$-dimensional spatial networks,
in which each node has a defined location $x_n$, and each edge has an associated 
energy barrier $\epsilon_{nm}$. The transition rates are then defined to be:
%
\begin{align}
w_{nm} \quad = \quad w_0 \eexp{-\epsilon_{nm}}B(x_n-x_m)
\end{align}
%
Where $B(r)$ describes the dependence of the coupling on the
distance between sites.


The model we start with has its sites distributed randomly and uniformly throughout
a $d$-dimensional hypercube, with $B(r) = \exp(-r/\xi)$. The disorder is defined by the location
of the sites, and its effect on the connectivity of the network can be manipulated through the 
parameter $\xi$. 



%%%%%%%%%%%%%%%%%%%%%%%%%%%%%%%%%%%%
\section{matrix categories}\label{sec:matrix_categories}

% Three categories:
% wnm = wmn  - no talk about steady state - equilibrium transient diffusion
%%            resnet
% wnm propto wmn - detailed balance - stochastic field
%           integral over it must be zero  boltzmann equil
%           should be added to the background
% wnm \ne wmn - assymetry - frustrated NESS

In general, the transition matrices have three symmetry categories:
\begin{itemize}
  \item 
    $w_{nm} = w_{mn}$ \ -\ In the symmetric case the system could reach
      equilibrium and ergodicity. The transient behavior might be diffusive. 
      This kind of system has the analog of a Resistor Network, which is often
      useful both analytically and numerically.
  \item
    $w_{nm} = \exp{-\frac{E_n-E_m}{T}} w_{mn}$\ -\ This is called detailed balance,
      and allows the system to reach Boltzmann's equilibrium. One could make
      the analogy of a "stochastic field", $\varepsilon(x) = \frac{E_n-E_m}{T}$.
      Any closed loop integral over such a field must equal zero.
      Note that any product of a symmetric operator with a diagonal operator
      can be said to satisfy detailed balance \cite{kologromov_zur_1936}
  \item
    $w_{nm} \ne w_{mn}$\ -\ In the assymetric case, a Non Equilibrium Steady
      State (NESS) might be reached. 
\end{itemize}

%%%%%%%%%%%%%%%%%%%%%%%%%%%%%%%%%%%%%%%%%%%%%%%%%%%%%
\section{TODO VRH}

%%%%%%%%%%%%%%%%%%%%%%%%%%%%%%%%%%%%%%%%%%%%
\section{Motivations for the model}
%% bullets - where does this model appear - balls and springs (leads to heat) - capacitors
%%             - conductance hopping

We have kept the model abstract and general on purpose, so that it
might fit a large number of physical situations. Some physical applications include:
%
\begin{itemize}
  \item
    Balls and Springs\ - \ A simple mechanical system of balls connected to each
      other by springs. The vibrational modes of the system are phonons, which conduct heat.
  \item
    Capacitors and Resistors\ -\ An electrical network of capacitors connected to each other
      by resistors.
  \item 
    Mott impurity hopping\ - \ This kind of conductance arises from hopping 
      between impurities.
\end{itemize}


%%%%%%%%%%%%%%%%%%%%%%%%%%%%%%%%%%%%%%%%%%%%%%%%%%%%%%%
\section{Relationship with Anderson localization}

In the Anderson model\cite{anderson_absense_1985}, a nearest neighbor network is formed
on a lattice. The bonds are all equal, and the on-site energies are random.
For $d=1$ and $d=2$, all states are localized, while for $d>2$, there exists
a transition between localized and extended modes. 

Localized eigenstates do not repel each other, and there

%%%%%%%%%%%%%%%%%%%%%%%%%%%%%%%%%%%%%%%%%%%%
\section{Modeling quantum diffusion with resistor networks}

In past work \cite{PRL_2000} 

In a model described by a Hamiltonian $\mathcal{H}$,
according to the first step of perturbation theory,
for very short time the transition probability is:
%
\begin{align}
WRONG p(n, n_0)\ =\ \left| \epsilon \bra{n} \mathcal{H}\ket{n_0}\right|^2
\end{align}
%


%%%%%%%%%%%%%%%%%%%%%%%%%%%%%%%%%%%%%%%%%%%%%%%%%%%%%
\section{About banded matrices}

Some facts about banded matrices, with 
\begin{align}
  w_{ij} = 
  \begin{cases}
    1 \quad &\textrm{  if  } |i-j|\le b   \\
    -\sum_{k \ne j} w_{kj} &\textrm{  if  } i = j  \\
    0 \quad &\textrm{  otherwise }
  \end{cases}
\end{align}

The analytical eigenvalues are:
\begin{align}
  \lambda &= 2\sum_{n=1..b} (\cos(n\cdot k) -1) \\
  k &= \frac{2\pi m}{N}
\end{align}


%%%%%%%%%%%%%%%%%%%%%%%%%%%%%%%%%%%%%%%%%%%%%%%%%%%%%%
\section{Sinai diffusion}

For example, by adding spatial correlations, the behavior can be changed 
from $S(t) \sim (\log t)^4$ to $S(t) \sim (\log t)^y$ with $y \in [4,\inf]$ \cite{stanley_generalization_1987}.

%%%%%%%%%%%%%%%%%%%%%%%%%%%%%%%%%%%%%%%%%%%%%%%%%%%%%%%
%\section{Organization}

%The organization of this proposal is in parts.
%\autoref{part1} includes this introduction,
%the objectives of the proposal and further background. 
%In \autoref{part2} we present the work accomplished so far.
%In \autoref{part3} lies the appendix, including some derivations,
%and an article accepted for publication at PRE, available in the arXiv \cite{de_leeuw_diffusion_2012}.
