%%%%%%%%%%%%%%%%%%%%%%%%%%%%%%%%%%%%%%%%%%%%%%%%%%%%%%%%%%%%%%%
%%%  notes
%%%%%%%%%%%%%%%%%%%%%%%%%%%%%%%%%%%%%%%%%%%%%%%%%%%%%%%%%%%%%%%

\documentclass[onecolumn,fleqn,notitlepage,secnumarabic]{revtex4}

% special 
\usepackage{ifthen}
\usepackage{ifpdf}
\usepackage{float}
\usepackage{color}

% fonts
\usepackage{latexsym}
\usepackage{amsmath} 
\usepackage{amssymb} 
\usepackage{bm}
\usepackage{wasysym}


\ifpdf
\usepackage{graphicx}
\usepackage{epstopdf}
\else
\usepackage{graphicx}
\usepackage{epsfig}
\fi

%by jarondl
\usepackage{subfig}
\usepackage{verbatim} % for multiline comments
\usepackage{natbib} 
\usepackage{fancybox}
\usepackage{cmap}  % for making pdf mathmode searchable
%\usepackage{sectsty}
\usepackage[pdftitle={Yaron de Leeuw's research proposal}]{hyperref}  % for hyperlinks in biblio. should be called last?

\graphicspath{{figures/},{PROG/figures/}}


%%%%%%%%%%%%%%%%%%%%%%%%%%%%%%%%%%%%%%%%%%%%%%%%%%%%%%%%%%%%%%%%

% NEW 
\newcommand{\abs}[1]{\left|#1\right|}
\newcommand{\varphiJ}{\bm{\varphi}}
\newcommand{\thetaJ}{\bm{\theta}}
%\renewcommand{\includegraphics}[2][0]{FIGURE}


% math symbols I
\newcommand{\sinc}{\mbox{sinc}}
\newcommand{\const}{\mbox{const}}
\newcommand{\trc}{\mbox{trace}}
\newcommand{\intt}{\int\!\!\!\!\int }
\newcommand{\ointt}{\int\!\!\!\!\int\!\!\!\!\!\circ\ }
\newcommand{\ar}{\mathsf r}
\newcommand{\im}{\mbox{Im}}
\newcommand{\re}{\mbox{Re}}

% math symbols II
\newcommand{\eexp}{\mbox{e}^}
\newcommand{\bra}{\left\langle}
\newcommand{\ket}{\right\rangle}

% Mass symbol
\newcommand{\mass}{\mathsf{m}} 
\newcommand{\rdisc}{\epsilon} 

% more math commands
\newcommand{\tbox}[1]{\mbox{\tiny #1}}
\newcommand{\bmsf}[1]{\bm{\mathsf{#1}}} 
\newcommand{\amatrix}[1]{\begin{matrix} #1 \end{matrix}} 
\newcommand{\pd}[2]{\frac{\partial #1}{\partial #2}}

% equations
\newcommand{\be}[1]{\begin{eqnarray}\ifthenelse{#1=-1}{\nonumber}{\ifthenelse{#1=0}{}{\label{e#1}}}}
\newcommand{\ee}{\end{eqnarray}} 

% arrangement
\newcommand{\hide}[1]{}
\newcommand{\drawline}{\begin{picture}(500,1)\line(1,0){500}\end{picture}}
\newcommand{\bitem}{$\bullet$ \ \ \ }
\newcommand{\Cn}[1]{\begin{center} #1 \end{center}}
\newcommand{\mpg}[2][1.0\hsize]{\begin{minipage}[b]{#1}{#2}\end{minipage}}
\newcommand{\mpgt}[2][1.0\hsize]{\begin{minipage}[t]{#1}{#2}\end{minipage}}



% extra math commands by jarondl
\newcommand{\inner}[2]{\left \langle #1 \middle| #2\right\rangle} % Inner product

%fminipage using fancybox package
\newenvironment{fminipage}%
  {\begin{Sbox}\begin{minipage}}%
  {\end{minipage}\end{Sbox}\fbox{\TheSbox}}


%%%%%%%%%%%%%%%%%%%%%%%%%%%%%%%%%%%%%%%%%%%%%%%%%%%%%%%%%%%%%%%%%%%%%%%%%%%

% Page setup
\setlength{\parindent}{0cm} 
\setlength{\parskip}{0.3cm} 

%%% Sections. The original revtex goes like this:
%\def\section{%
%  \@startsection
%    {section}%
%    {1}%
%    {\z@}%
%    {0.8cm \@plus1ex \@minus .2ex}%
%    {0.5cm}%
%    {\normalfont\small\bfseries}%
%}%
%\def\subsection{%
%  \@startsection
%    {subsection}%
%    {2}%
%    {\z@}%
%    {.8cm \@plus1ex \@minus .2ex}%
%    {.5cm}%
%    {\normalfont\small\bfseries}%
%}%
%%%%%%% And our version goes like this:
\makeatletter
\def\section{%
  \@startsection
    {section}%
    {1}%
    {\z@}%
    {0.8cm \@plus1ex \@minus .2ex}%
    {0.5cm}%
    {\Large\bf $=\!=\!=\!=\!=\!=\;$}%
}%
\def\subsection{%
  \@startsection
    {subsection}%
    {2}%
    {\z@}%
    {.8cm \@plus1ex \@minus .2ex}%
    {.5cm}%
    {\normalfont\small\bfseries$=\!=\!=\!=\;$}%
}%
%%%%%%%%%%  Here we deal with capitalization. The original revtex first, and then our version.
%\def\@hangfrom@section#1#2#3{\@hangfrom{#1#2}\MakeTextUppercase{#3}}%
%\def\@hangfroms@section#1#2{#1\MakeTextUppercase{#2}}%
\def\@hangfrom@section#1#2#3{\@hangfrom{#1#2}{#3}}%
\def\@hangfroms@section#1#2{#1{#2}}%
\makeatother


%%%%%%%%%%%%%%%%%%%%%%%%%%%%%%%%%%%%%%%%%%%%%%%%%%%%%%%%%%%%%
%%%%%%%%%%%%%%%%%%%%%%%%%%%%%%%%%%%%%%%%%%%%%%%%%%%%%%%%%%%%%
\begin{document}

\title{Diffusion properties of mesoscopic systems}

\author{Yaron de Leeuw, Doron Cohen}
\date{\today}
\maketitle


%%%%%%%%%%%%%%%%%%%%%%%%%%%%%%%%%%%%%%%%%%%%%%%%%%%%%%%%%%%%%%%%%%%%%%%%%%%%%%%%%%%%%%%%%%%%
%%%%%%%%%%%%%%%%%%%%%%%%%%%%%%%%%%%%%%%%%%%%%%%%%%%%%%%%%%%%%%%%%%%%%%%%%%%%%%%%%%%%%%%%%%%

\section{Introduction}
\subsection{Defining site occupation probabilities and transition rates}
First let us define the model. The model consists of sites which a particle may occupy. At each point in time, and for every site, there is a probability to find the particle in this particular site, which is called \emph{occupation probability}. We define the \emph{transition rates} as the rate in which those probabilites move from one site to another.

\subsection{Rate equation}
We use $P_n$ to denote the $n$'th site occupation probability, and $W_{nm}$ for the transition rate between two sites $n$ and $m$. The time evolution of $P_n$ may be written as:
\begin{align}
\frac{dP_n(t)}{dt} = \sum_m W_{nm}P_m(t)
\end{align}
Because of probability conservation, we want to have $\sum_m W_{nm} = 0 \  \forall m$ , which we can achieve by setting $W_{nn} = -\sum_{m\ne n} W_{nm} $, meaning that for each site the sum of incoming transition rates negates the outgoing transitions.

In a vector form, we have:
\begin{align} \boldsymbol{ \dot P } = \boldsymbol{W} \boldsymbol{P} \end{align}

\subsection{Finding the values for the the transition rates}
Most of the systems we will work with can be defined unambiguously by defining their transition rates. We can map certain physical, geometrical or topological charecteristics of systems to their transition rate matrices. The transition rates may, for example, depened on the distance betweeen scattered points, in either $1d$, $2d$, or any other dimension.



\subsection{..}
 We may also define an energy for each site, contributing via Boltzmann's factor to the transition rates.


%\bibliographystyle{plainnat}
\bibliography{jarondl}
\end{document}
