%%%%%%%%%%%%%%%%%%%%%%%%%%%%%%%%%%%%%%%%%%%%%%%%%%%%%%%%%%%%%%%
%%%  dts notes
%%%%%%%%%%%%%%%%%%%%%%%%%%%%%%%%%%%%%%%%%%%%%%%%%%%%%%%%%%%%%%%

\documentclass[onecolumn,fleqn]{revtex4}

% special 
\usepackage{ifthen}
\usepackage{ifpdf}
\usepackage{float}
\usepackage{color}

% fonts
\usepackage{latexsym}
\usepackage{amsmath} 
\usepackage{amssymb} 
\usepackage{bm}
\usepackage{wasysym}

\ifpdf
\usepackage{graphicx}
\usepackage{epstopdf}
\else
\usepackage{graphicx}
\usepackage{epsfig}
\fi

% extra by jarondl
\usepackage[pdftitle={DTS},bookmarks]{hyperref}
\usepackage{array}

%%%%%%%%%%%%%%%%%%%%%%%%%%%%%%%%%%%%%%%%%%%%%%%%%%%%%%%%%%%%%%%%

% NEW 
\newcommand{\abs}[1]{\left|#1\right|}
\newcommand{\varphiJ}{\bm{\varphi}}
\newcommand{\thetaJ}{\bm{\theta}}
%\renewcommand{\includegraphics}[2][0]{FIGURE}
\newcommand{\rmrk}[1]{\textcolor{red}{#1}}
\newcommand{\Eq}[1]{\textcolor{blue}{Eq.\!\!~(\ref{#1})}} 
\newcommand{\Fig}[1]{\textcolor{blue}{Fig.}\!\!~\ref{#1}}

% math symbols I
\newcommand{\sinc}{\mbox{sinc}}
\newcommand{\const}{\mbox{const}}
\newcommand{\trc}{\mbox{trace}}
\newcommand{\intt}{\int\!\!\!\!\int }
\newcommand{\ointt}{\int\!\!\!\!\int\!\!\!\!\!\circ\ }
\newcommand{\ar}{\mathsf r}
\newcommand{\im}{\mbox{Im}}
\newcommand{\re}{\mbox{Re}}

% math symbols II
\newcommand{\eexp}{\mbox{e}^}
\newcommand{\bra}{\left\langle}
\newcommand{\ket}{\right\rangle}

% Mass symbol
\newcommand{\mass}{\mathsf{m}} 
\newcommand{\rdisc}{\epsilon} 

% more math commands
\newcommand{\tbox}[1]{\mbox{\tiny #1}}
\newcommand{\bmsf}[1]{\bm{\mathsf{#1}}} 
\newcommand{\amatrix}[1]{\begin{matrix} #1 \end{matrix}} 
\newcommand{\pd}[2]{\frac{\partial #1}{\partial #2}}

% equations
\newcommand{\mylabel}[1]{\label{#1}} 
\newcommand{\beq}{\begin{eqnarray}}
\newcommand{\eeq}{\end{eqnarray}} 
\newcommand{\be}[1]{\begin{eqnarray}\ifthenelse{#1=-1}{\nonumber}{\ifthenelse{#1=0}{}{\mylabel{e#1}}}}
\newcommand{\ee}{\end{eqnarray}} 

% arrangement
\newcommand{\hide}[1]{}
\newcommand{\drawline}{\begin{picture}(500,1)\line(1,0){500}\end{picture}}
\newcommand{\bitem}{$\bullet$ \ \ \ }
\newcommand{\Cn}[1]{\begin{center} #1 \end{center}}
\newcommand{\mpg}[2][1.0\hsize]{\begin{minipage}[b]{#1}{#2}\end{minipage}}
\newcommand{\mpgt}[2][1.0\hsize]{\begin{minipage}[t]{#1}{#2}\end{minipage}}

%%%%%%%%%%%%%%%%%%%%%%%%%%%%%%%%%%%%%%%%%%%%%%%%%%%%%%%%%%%%%%%%%%%%%%%%%%%

% Page setup
\setlength{\parindent}{0cm} 
\setlength{\parskip}{0.3cm} 

%counters
\renewcommand{\thesection}{\arabic{section}}
\renewcommand{\thesubsection}{\arabic{subsection}}
\setcounter{section}{0}
\setcounter{subsection}{0}

% Sections
\newcommand{\sect}[1]
{
\addtocounter{section}{1} 
\setcounter{subsection}{0}
\ \\ 
\pdfbookmark[2]{\thesection. \ #1}{sect.\thesection}
{\Large\bf $=\!=\!=\!=\!=\!=\;$ [\thesection] \ #1}  
\nopagebreak
}

% subections
\newcommand{\subsect}[1]
{
\addtocounter{subsection}{1} 
\ \\ 
\pdfbookmark[2]{\ \ \ \ \thesection.\thesubsection. \ #1}{subsect.\thesection.\thesubsection}
{\bf $=\!=\!=\!=\!=\!=\;$ [\thesection.\thesubsection] \ #1}  
\nopagebreak
}

%%%%%%%%%%%%%%%%%%%%%%%%%%%%%%%%%%%%%%%%%%%%%%%%%%%%%%%%%%%%%%%%%%%%%%%%%%%

%\renewcommand{\includegraphics}[2][]{\ \\ \ {\color{blue} FIGURE:} \ \\ \ }
\graphicspath{{new/}}

%%%%%%%%%%%%%%%%%%%%%%%%%%%%%%%%%%%%%%%%%%%%%%%%%%%%%%%%%%%%%%%%%%%%%%%%
%%%%%%%%%%%%%%%%%%%%%%%%%%%%%%%%%%%%%%%%%%%%%%%%%%%%%%%%%%%%%
\begin{document}

\title{Resistor network modeling of diffusion in sparse networks}

\author{DC, notes for BGU-WES project}

\maketitle

%%%%%%%%%%%%%%%%%%%%%%%%%%%%%%%%%%%%%%%%%%%%%%%%%%%%%%%%%%%%%%%%%%%%%%%%
%%%%%%%%%%%%%%%%%%%%%%%%%%%%%%%%%%%%%%%%%%%%%%%%%%%%%%%%%%%%%%%%%%%%%%%%


%%%%%%%%%%%%%%%%%%%%%%%%%%%%%%%%%%%%%%%%%%%%%%%%%%%%%%%%%%%
\sect{The model}

We intend to study an extremely simple model 
that is described by the Hamiltonian matrix
%
\beq
\mathcal{H} \ \ = \ \ \text{diag}\{E_n\} + \epsilon \{B_{nm}\}
\eeq
%
where $B_{nm}$ are the elements of a real symmetric matrix  
%
\beq
B_{nm} \ \ = \ \ \text{random}[\pm] \ \eexp{-\text{random}[x]}, 
\ \ \ \ \ \ \ \ \ 0 < |n-m| \le b    
\eeq
%
We assume that the entries $x$ are random numbers 
with box (uniform) distribution within an interval $[0,\sigma]$. 
Based on the definition of KBD Eq(33) one obtains Eq(D5) there 
that can be written as 
%
\beq
s \ \ = \ \ \frac{2}{\sigma} \tanh\left(\frac{\sigma}{2}\right)
\eeq
 
The above Hamiltonian can describe the motion of a particle 
in a disordered potential. We assume that $E_n=0$, which is like 
having no electric field. 
What we call "sparsity" in our language 
means "off diagonal disorder" with log-wide distribution. 
The limit $\sigma\rightarrow0$ means no sparsity 
(Bloch lattice with sign randomization). 

Since we do not have "electric field" we can set the units 
of time such that ${\epsilon=1}$.
Thus the model is defined by just two dimensionless parameters:
%
\beq
b \ \  &=& \ \ \text{bandwidth} \\
\sigma  \ \  &=& \ \ \text{sparsity} 
\eeq
%
It make sense to conjecture that there is only 
one time scale in the dynamics: 
%
\beq
t_{b,\sigma}  \ \ = \ \  \frac{\tau(b,\sigma)}{\epsilon}  
\eeq
%
Such that the transient diffusion coefficient is 
%
\beq
D_n  \ \ = \ \  \frac{b^2 \ \epsilon}{\tau}  
\eeq
%
This conjecture has been established in our past work for regular disorder.



%%%%%%%%%%%%%%%%%%%%%%%%%%%%%%%%%%%%%%%%%%%%%%%%%%%%%%%%%%%
\sect{Past work}

Let us recall the reasoning of our past work [PRL 2000]. 
From perturbation theory it follows that for very short time  
the probability for a transition is 
%
\beq
p(n_0\leadsto n) \ \  =  \ \ |\epsilon B_{n,n_0} \ t|^2
\eeq
%
Hence the total probability for a transition 
from an initial state is 
%
\beq
p(t) \ \  \sim \ \  b \times \epsilon^2  \ t^2  \qquad \qquad\qquad\qquad \mbox{\rmrk{[This had a typo, with $p(t) \sim b^2\times\epsilon^2t^2$ ]}}
\eeq
%


The time scale $t_{b}$ is determined by the 
condition ${p(t)\sim 1}$ leading to 
%
\beq
t_b \ \ \sim \ \  \frac{1}{b^{1/2} \ \epsilon}  
\eeq
%
and consequently 
%
\beq
D_n \ \ \sim \ \  b^{5/2} \ \epsilon   
\eeq

%%%%%%%%%%%%%%%%%%%%%%%%%%%%%%%%%%%%%%%%%%%%%%%%%%%%%%%%%%%

\sect{Conjectures}

Once we have a distribution 
one should replace $\epsilon$ by some effective value. 
The natural tendency is to use 
%
\beq
\epsilon \ \ \mapsto \ \ \text{RMS}[B_{nm}] \ \epsilon
\eeq 
%
Which means that our way to calculate $D$ in the past was 
%
\beq
D_n  \ \ = \ \ \Big[b \ \text{Var}[B_{nm}] \Big]^{1/2}   \ b^2 \ \epsilon 
\eeq 
%
But for log-wide distribution this RMS value is expected 
to be suppressed. We expect the suppression factor to be 
determined by a resistor network calculations.
We shall write the expression for the diffusion coefficient as 
%
\beq
D_n  \ \ = \ \ \frac{1}{\tau(b,\sigma)}  \ b^2 \ \epsilon 
\eeq 
%
where $\tau$ is the dimensionless time scale 
that characterizes the dynamics. 
Again we emphasize: we conjecture that there is only 
one time scale is this problem.

%%%%%%%%%%%%%%%%%%%%%%%%%%%%%%%%%%%%%%%%%%%%%%%%%%%%%%%%%%%

\sect{Challenges}

The challenge is to establish that 
for large $\sigma$ the dimensionless break time 
becomes longer than $\tau=b^{-1/2}$, 
and to figure out how it depends on $\sigma$. 
On the analytical side we hope to 
establish a relation between $1/\tau$ 
and the resisitivity of the network $G_{nm}=|B_{nm}|^2$.


%%%%%%%%%%%%%%%%%%%%%%%%%%%%%%%%%%%%%%%%%%%%%%%%%%%%%%%%%%%

\sect{Numerics}

Say $b=20$. We make simulations for various values 
of $\sigma$. First we confirm, for debugging purpose, 
the "old recipe" that should hold for small $\sigma$, 
then we go to large $\sigma$. The outputs are as usual: 
%
\beq
P(t) \ \ &=&  \ \ \text{The survival probability} \\
S_{N}(t) \ \ &=& \ \ \text{The participation number} \\
S_{50\%}(t) \ \ &=& \ \ \text{The 50\% width of the distribution} \\
S(t) \ \ &=& \ \ \text{The second-moment based spreading} 
\eeq


%%%%%%%%%%%%%%%%%%%%%%%%%%%%%%%%%%%%%%%%%%%%%%%%%%%%%%%%%%%

\sect{Extensions}

Natural extension is to add the ``electric field" and to discuss ``regimes". 
However, the main motivation is to use later these results 
for the analysis of "quantum stochastic diffusion". 




%%%%%%%%%%%%%%%%%%%%%%%%%%%%%%%%%%%%%%%%%%%%%%%%%%%%%%%%%%%

\sect{Results}

The RMS value of the elements of B is calculated as follows:
%
\begin{align}
\textrm{VAR}(B)\quad &=\quad \left\langle  B_{nm}^2 \right\rangle  \quad =
 \quad  \int_0^\sigma \quad \left( \pm\eexp{-x}\right)^2 \frac{dx}{\sigma} 
= \quad  \frac{1-\eexp{-2\sigma}}{2\sigma} &\\
\textrm{RMS}(\sigma) \quad &= \quad \sqrt{\textrm{VAR}(B)}  \quad  =  \quad  \sqrt{ \frac{1-\eexp{-2\sigma}}{2\sigma}  }
\end{align}
%

The linear approximation is:
%
\beq
D_\textrm{linear} \quad=\quad b^{2.5} \quad \textrm{RMS}(\sigma)\quad\epsilon
\eeq
%
In \ref{fig:D}, \ref{fig:v_square} we plot:
%
\begin{align}
\textrm{scaled} \left[ D\right] \quad &=  \quad \frac{D}{D_\textrm{linear}} \quad=
\quad  \frac{D}{b^{2.5} \quad \textrm{RMS} \quad \epsilon } \\[\bigskipamount]
\textrm{scaled} \left[ v^2\right] \quad &= \quad \frac{v^2}{b^3 \quad \textrm{RMS}^2 \quad \epsilon^2 } \\[\bigskipamount]%\textrm{scaled} \quad t_\textrm{bal} \quad &= \quad t_\textrm{bal} \quad b^\frac{1}{2} \quad \textrm{RMS} \epsilon
\end{align}
%

In \ref{fig:D2_vs_gs} we plot the scaled $D_2$ vs $g_s(\sigma, b)$, where the suppression factor 
$g_s(\sigma,b)$ is:
% 
\beq
g_s(\sigma,b) = \frac{\left(1+\frac{n_c}{2b}\right)\eexp{-\frac{n_c}{2b}\sigma} - \eexp{-2\sigma}}{1-\eexp{-2\sigma}}
\eeq
%
With $n_c \approx 2$.



We find that that 
\begin{align}
\textrm{scaled}\left[D\right] \approx \frac{1}{2}g_s(\sigma,b)
\end{align}
 up to some factors, meaning 
there is suppression due to sparsity effects.



%%%%%%%%%%%%%%%%%%%%%%%%%%%%%%%%%%%%%%%%%%%%%%%%%%%%%%%%%%%%%%
\sect{Plots}

$D$ was calculated by two methods:

$D_1$ is the "diffusion
coefficient measured by a power law fitting with 
tolerance $D\times t^{\alpha}\quad ;\quad \alpha>0.9$ " 

$D_2$ the variance is devided by time, evaluated at the plato point.

\begin{figure}[H]
\includegraphics[height=0.4\textheight]{D1} \\
\includegraphics[height=0.4\textheight]{D2} 
\caption{ These are the scaled $D_1$ and $D_2$, as a function of $\sigma$. 
    Both present proper dependance on the sparsity, but $D_2$ seems more consistent.}\label{fig:D}
\end{figure}


\begin{figure}[H]
\includegraphics[height=0.4\textheight]{D2_vs_gs} 
\caption{ This is $D_2$ vs $g_s(\sigma)$. The black segmented line has a slope of $1$ while the gray dotted line has a slope of $0.5$}\label{fig:D2_vs_gs}
\end{figure}

\begin{figure}[H]
\includegraphics[height=0.4\textheight]{v_square} 
\caption{ The scaled $v^2$, as a function of $\sigma$.}\label{fig:v_square}
\end{figure}

\end{document}
