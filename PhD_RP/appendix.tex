\chapter{Published Papers}\label{sec:papers}

\includepdf[pages=-]{1206_2495v2}
%%%%%%%%%%%%%%%%%%%%%%%%%%%%%%%%%%%%%%%%%%%%%%
%%%%%%%%%%%%%%%%%%%%%%%%%%%%%%%%%%%%%%%%%%%%%%
\chapter{Spacing Statistics in $d$-dimensions}

Some clarifing points regarding the statistics of uniformly distributed sites 
will be made in this section.


There are $N$ sites distributed randomly within a $d$-dimensional
hypercube of volume $L^d$. We define a typical length $r_0$ by:
%
\begin{align}
\frac{L^d}{r_0^d} \ =\ N
\end{align}
%
From now on we shall ignore boundary conditions, assuming $N$ is large enough.
In the numerics we have used periodic boundary conditions.


If we choose some arbitrary point as the origin, the distribution of sites
around this point will be:
%
\begin{align}\label{eq:rho}
\rho(r)dr = \frac{\Omega_d r^{d-1}}{r_0^d}dr
\end{align}
% 
Where $\Omega_d$ is the $d$ dimensional solid angle:
%
\begin{align}
\Omega_d \ =\ 2,\ 2\pi,\ 4\pi, \ldots
\end{align}
%

The distribution of the nearest neighbor distance can be 
derived \cite{hertz_uber_1909,*Chandrasekhar_stochastic_1943,*Torquato_nearest-neighbor_1990},
using a differential equation. 
\rmrk{(Apparently my previous extension of the $1d$ solution to $d$ dimensions
was just a good guess, the proof is a bit less trivial then I've thought)}. %TODO remove this remark

Let us denote by $P(r)$ the probability that the first near neighbor will be
between $r$ and $r+dr$. The probability of having no neighbors up to distance $r$ is
%
\begin{align}
P_0(r) \ =\  1 - \int_0^r P(r)dr
\end{align}
$P(r)$ must equal the probability of having no neighbors up to distance $r$ times
the probabilty of finding a neighbor between $r$ and $r+dr$. So $P(r)$ must satisfy:
%
\begin{align}
P(r) \ &=\ P_0(r) \times \rho(r) \ =\ \left[ 1-\int_0^r P(r)dr \right] \rho(r) \\
\frac{P(r)}{\rho(r)}\ &=\  1-\int_0^r P(r)dr \\
\frac{d}{dr}\left(\frac{P(r)}{\rho(r)}\right) \ &=\ -\rho(r)\frac{ P(r)}{\rho(r)}
\end{align}
%
Which has the solution:
%
\begin{align}
P(r) \ &=\ \rho(r) \eexp{-\int_0^r \rho(r) dr} \\
       &=\ \frac{\Omega_d r^{d-1}}{r_0^d} \eexp{-\frac{\Omega_d}{d} \left(\frac{r}{r_0}\right)^d}
\end{align}
%
Where in the last step we have plugged in $\rho(r)$ from \autoref{eq:rho}
%


%The probability to have an empty surface of radius $r$ and thickness $\Delta r$
%surrounding the origin is 
%
%\begin{align}
%P_0(r<x<r+\Delta r) = (1-\rho(r)\Delta r)
%\end{align}
%
%To have an empty hypersphere, we need to have all of the surfaces empty. 
%So we can formally write $r=N\Delta r$, and multiply the probabilities:
%
%\begin{align}
%P_0(r) = \lim_{N\to \infty} \left( 1- \rho(N\Delta r)\right)^N
%\end{align}
%


%%%%%%%%%%%%%%%%%%%%%%%%%%%%%%%%%%%%%%%%%%%%%%
%%%%%%%%%%%%%%%%%%%%%%%%%%%%%%%%%%%%%%%%%%%%%%
\chapter{Resistor Network Calculation}




