%%%%%%%%%%%%%%%%%%%%%%%%%%%%%%%%%%%%%%%%%%%%%%%%%%%%%%%%%%%%%%%
%%%  pts
%%%%%%%%%%%%%%%%%%%%%%%%%%%%%%%%%%%%%%%%%%%%%%%%%%%%%%%%%%%%%%%

\documentclass[onecolumn,fleqn,12pt,openany,a4paper,longbibliography,oneside]{book}

%%%%%%%%%%
%% This is for usage with XELATEX instead of latex

\renewcommand{\baselinestretch}{1.5} 
\topmargin -1.5cm      
\oddsidemargin -0.04cm   
\evensidemargin -0.04cm  
\textwidth 16.59cm
\textheight 24cm 
\pagestyle{empty}  % no page numbers

% special 
\usepackage{ifthen}
\usepackage{ifpdf}
\usepackage{color}

\ifpdf
\usepackage{graphicx}
\usepackage{epstopdf}
\else
\usepackage{graphicx}
\usepackage{epsfig}
\fi

% fonts
\usepackage{latexsym}
\usepackage{amsmath}
\usepackage{amssymb}
\usepackage{bm}
\usepackage{wasysym}

%by jarondl
%%\usepackage{wordlike}
\usepackage{subcaption}
\usepackage[numbers,sort&compress,elide]{natbib}
% merge - {feynmann,*salam,*epr} are under one number
% elide - same as megre, but common parts appear once.
%
\usepackage[draft]{pdfpages}
% draft or final
\usepackage[nottoc]{tocbibind}
% addes the biblio to toc

%%%%%%%%%%%%%
%%% Hebrew support for hebrew abstract:
%\usepackage[utf8]{inputenc}
%\usepackage[hebrew,english]{babel}
%\usepackage{bidi}
\usepackage{polyglossia}
\setdefaultlanguage{hebrew}
\setotherlanguage{english}
%\setmainfont{David CLM}
\newfontfamily\hebrewfont[Script=Hebrew]{David CLM}

%\usepackage[colorlinks=true,pagebackref=true,pdftitle={Research Proposal}]{hyperref}
\graphicspath{{figures/}}

%%%%%%%%%%%%%%%%%%%%%%%%%%%%%%%%%%%%%%%%%%%%%%%%%%%%%%%%%%%%%%%%

% NEW 
\newcommand{\abs}[1]{\left|#1\right|}
\newcommand{\Prob}{\mbox{Prob}\,}
\newcommand{\erf}{\mbox{erf}\,}
\newcommand{\df}{{\rm d}}

% math symbols I
\newcommand{\sinc}{\mbox{sinc}}
\newcommand{\const}{\mbox{const}}
\newcommand{\trc}{\mbox{trace}}
\newcommand{\intt}{\int\!\!\!\!\int }
\newcommand{\ointt}{\int\!\!\!\!\int\!\!\!\!\!\circ\ }
\newcommand{\ar}{\mathsf r}
\newcommand{\im}{\mbox{Im}}
\newcommand{\re}{\mbox{Re}}

% math symbols II
\newcommand{\eexp}{\mbox{e}^}
\newcommand{\bra}{\left\langle}
\newcommand{\ket}{\right\rangle}

% Mass symbol
\newcommand{\mass}{\mathsf{m}} 
\newcommand{\Mass}{\mathsf{M}} 

% more math commands
\newcommand{\tbox}[1]{\mbox{\tiny #1}}
\newcommand{\bmsf}[1]{\bm{\mathsf{#1}}} 
%\newcommand{\amatrix}[1]{\matrix{#1}} 
\newcommand{\amatrix}[1]{\begin{matrix} #1 \end{matrix}} 
\newcommand{\pd}[2]{\frac{\partial #1}{\partial #2}}

% equations
\newcommand{\mylabel}[1]{\label{#1}} 
%\newcommand{\mylabel}[1]{\textcolor{blue}{[#1]}\label{#1}} 
\newcommand{\beq}{\begin{eqnarray}}
\newcommand{\eeq}{\end{eqnarray}} 
\newcommand{\be}[1]{\begin{eqnarray}\ifthenelse{#1=-1}
{\nonumber}{\ifthenelse{#1=0}{}{\mylabel{e#1}}}}
\newcommand{\ee}{\end{eqnarray}} 

% arrangement
\newcommand{\drawline}{\begin{picture}(500,1)\line(1,0){500}\end{picture}}
\newcommand{\bitem}{$\bullet$ \ \ \ }
\newcommand{\Cn}[1]{\begin{center} #1 \end{center}}
\newcommand{\mpg}[2][1.0\hsize]{\begin{minipage}[b]{#1}{#2}\end{minipage}}
\newcommand{\mpgt}[2][1.0\hsize]{\begin{minipage}[t]{#1}{#2}\end{minipage}}
\newcommand{\putgraph}[2][width=0.30\hsize]{\includegraphics[#1]{#2}}

% more
%\newcommand{\Eq}[1]{Eq.\!\!~(\ref{#1})}
%\newcommand{\Fig}[1]{Fig.\!\!~\ref{#1}}  
\newcommand{\Eq}[1]{\textcolor{blue}{Eq.\!\!~(\ref{#1})}} 
\newcommand{\Fig}[1]{\textcolor{blue}{Fig.}\!\!~\ref{#1}} 
\newcommand{\hide}[1]{}
%\newcommand{\hide}[1]{\textcolor{red}{[hidden text]}} %{}
\newcommand{\rmrk}[1]{\textcolor{red}{#1}}
%\newcommand{\Rmrk}[1]{\textcolor{blue}{\LARGE\bf #1}}

%\renewcommand{\includegraphics}[2][]{\ \\ \ FIGURE: \ \\ \ }
%\renewcommand{\cite}[1]{\textcolor{blue}{[\onlinecite{#1}}]} %{[\onlinecite{#1}]} 


%%%%%%%%%%%%%%%%%%%%%%%%%%%%%%%%%%%%%%%%%%%%%%%%%%%%%%%%%%%%%
%%%%%%%%%%%%%%%%%%%%%%%%%%%%%%%%%%%%%%%%%%%%%%%%%%%%%%%%%%%%%
\begin{document}
{\setlength{\parindent}{0cm}
\begin{center}

{\Huge\bfseries
אוניברסיטת בן גוריון בנגב


הצעת תוכנית מחקר ללימודי דוקטורט
\vspace{6em}



שרידות וניידות במערכות דלילות מזוגגות




\begin{english}

Survival and Transport in Glassy Sparse Systems

\vspace{2em}

Yaron de Leeuw
\end{english}

ירון דה ליאו


31.10.2012

}% ending the "large"
\end{center}
\vspace{8em}

חתימת מנחה: 
\underline{\hspace{20ex}}


חתימת יו"ר ועדת מוסמכים מחלקתי:
\underline{\hspace{20ex}}
}%ending the no indet 
\newpage
\section*{תקציר}
אנו חוקרים הסעה והתפשטות ברשתות דלילות לא מסודרות. בתחילה, לכל אתר משויך מיקום מרחבי. 
אנו משתמשים במטריצת המרחקים בין הנקודות כבסיס להגדרת משוואות קצב או המילטוניאנים.
בנוסף לכך, המטריצה יכולה להיות תלויה במשתנה צימוד המשויך לקשתות (החיבורים שבין האתרים).
לפיכך, על ידי הגרלת מיקום האתרים או משתני הצימוד, אנו יכולים לגרום לאי סדר במערכת.
אם המשתנים השונים בבעיה שונים בסדרי גודל, כאשר רובם קטן מאוד ביחס לקבוצה מצומצמת, 
ניתן לקרוא לרשת דלילה. תחום העניין המרכזי שלנו הוא תכונות ההסעה ארוכות הטווח במערכות דלילות אלה.


מטרתנו במחקר זה היא קישור של מספר גישות מקובלות לחקר ההסעה,
ובפרט משל רשת הנגדים, הפיתוח הספקטרלי של השרידות
ושיטת מטריצות השונות המשותפת לחקר הולכת חום.
יש סיבות להאמין כי גישות אלה הן למעשה צדדים שונים
של אותו רעיון בסיסי, ואמונתנו היא שהבנת הקשר בין השיטות
תוביל להבנה מעמיקה יותר בכל שיטה כשלעצמה.


במערכת סטוכסיטיות, אנו
רואים כי במימד אחד הדלילות גוררת תת-דיפוזיה,
בעוד במימדים גבוהים יותר עדיין נשמרת הדיפוזיה הרגילה,
אך עם מקדם דיפוזיה נמוך יותר.
על ידי שילוב רעיונות משיטת "דילוג הטווח המשתנה" 
\begin{english}
(VRH)
\end{english}
ורעיונות מתחום תורת הפרקולציה,
הצענו שיטה בשם "דילוג הטווח האפטקיבי"
\begin{english}
(ERH)
\end{english}
 על מנת להעריך את מקדם הדיפוזיה.
אנו משווים תוצאות חישוביות למקדם הדיפוזיה עם הקירוב הלינארי ועם קירוב "דילוג הטווח האפטקיבי".



אנו דנים בהרחבות נוספות, ובכללן מספר מרובה של חלקיקים, 
הרחבת התוצאות לדיפוזיה קוונטית ולמטריצות לא סימטריות
וגרסא דו מימדית למודל סיני
\begin{english}
.(Sinai)
\end{english}


\end{document} 


