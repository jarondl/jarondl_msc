%%%%%%%%%%%%%%%%%%%%%%%%%%%%%%%%%%%%%%%%%%%%%%%%%%%%%%%%%%%%%%%
%%%  dts notes
%%%%%%%%%%%%%%%%%%%%%%%%%%%%%%%%%%%%%%%%%%%%%%%%%%%%%%%%%%%%%%%

\documentclass[onecolumn,fleqn]{revtex4}

% fonts
\usepackage{latexsym}
\usepackage{amsmath} 
\usepackage{amssymb} 
\usepackage{bm}
\usepackage{wasysym}

\usepackage{graphicx}


% extra by jarondl
\usepackage{array}
%\usepackage{multicol}
\usepackage[caption=false]{subfig} %subcaption is not compat with revtex
\usepackage[pdftitle={DTS},bookmarks]{hyperref}


%%%%%%%%%%%%%%%%%%%%%%%%%%%%%%%%%%%%%%%%%%%%%%%%%%%%%%%%%%%%%%%%

% NEW 
\newcommand{\abs}[1]{\left|#1\right|}
\newcommand{\varphiJ}{\bm{\varphi}}
\newcommand{\thetaJ}{\bm{\theta}}
%\renewcommand{\includegraphics}[2][0]{FIGURE}
\newcommand{\rmrk}[1]{\textcolor{red}{#1}}
\newcommand{\Eq}[1]{\textcolor{blue}{Eq.\!\!~(\ref{#1})}} 
\newcommand{\Fig}[1]{\textcolor{blue}{Fig.}\!\!~\ref{#1}}

% math symbols I
\newcommand{\sinc}{\mbox{sinc}}
\newcommand{\const}{\mbox{const}}
\newcommand{\trc}{\mbox{trace}}
\newcommand{\intt}{\int\!\!\!\!\int }
\newcommand{\ointt}{\int\!\!\!\!\int\!\!\!\!\!\circ\ }
\newcommand{\ar}{\mathsf r}
\newcommand{\im}{\mbox{Im}}
\newcommand{\re}{\mbox{Re}}

% math symbols II
\newcommand{\eexp}{\mbox{e}^}
\newcommand{\bra}{\left\langle}
\newcommand{\ket}{\right\rangle}

% Mass symbol
\newcommand{\mass}{\mathsf{m}} 
\newcommand{\rdisc}{\epsilon} 

% more math commands
\newcommand{\tbox}[1]{\mbox{\tiny #1}}
\newcommand{\bmsf}[1]{\bm{\mathsf{#1}}} 
\newcommand{\amatrix}[1]{\begin{matrix} #1 \end{matrix}} 
\newcommand{\pd}[2]{\frac{\partial #1}{\partial #2}}

% equations
\newcommand{\mylabel}[1]{\label{#1}} 
\newcommand{\beq}{\begin{eqnarray}}
\newcommand{\eeq}{\end{eqnarray}} 
\newcommand{\be}[1]{\begin{eqnarray}\ifthenelse{#1=-1}{\nonumber}{\ifthenelse{#1=0}{}{\mylabel{e#1}}}}
\newcommand{\ee}{\end{eqnarray}} 

% arrangement
\newcommand{\hide}[1]{}
\newcommand{\drawline}{\begin{picture}(500,1)\line(1,0){500}\end{picture}}
\newcommand{\bitem}{$\bullet$ \ \ \ }
\newcommand{\Cn}[1]{\begin{center} #1 \end{center}}
\newcommand{\mpg}[2][1.0\hsize]{\begin{minipage}[b]{#1}{#2}\end{minipage}}
\newcommand{\mpgt}[2][1.0\hsize]{\begin{minipage}[t]{#1}{#2}\end{minipage}}

%%%%%%%%%%%%%%%%%%%%%%%%%%%%%%%%%%%%%%%%%%%%%%%%%%%%%%%%%%%%%%%%%%%%%%%%%%%
% Sections
\newcommand{\sect}[1]
{
\addtocounter{section}{1} 
\setcounter{subsection}{0}
\ \\ 
\pdfbookmark[2]{\thesection. \ #1}{sect.\thesection}
{\Large\bf $=\!=\!=\!=\!=\!=\;$ [\thesection] \ #1}  
\nopagebreak
}

% subections
\newcommand{\subsect}[1]
{
\addtocounter{subsection}{1} 
\ \\ 
\pdfbookmark[2]{\ \ \ \ \thesection.\thesubsection. \ #1}{subsect.\thesection.\thesubsection}
{\bf $=\!=\!=\!=\!=\!=\;$ [\thesection.\thesubsection] \ #1}  
\nopagebreak
}
%%%%%%%%%%%%%%%%%%%%%%%%%%%%%%%%%%%%%%%%%%%%%%%%%%%%%%%%%%%%%%%%%%%%%%%%
%%%%%%%%%%%%%%%%%%%%%%%%%%%%%%%%%%%%%%%%%%%%%%%%%%%%%%%%%%%%%

\graphicspath{{figures/}}
\begin{document}

\title{PTA}

\author{YdL}

\maketitle

%%%%%%%%%%%%%%%%%%%%%%%%%%%%%%%%%%%%%%%%%%%%%%%%%%%%%%%%%%%%%%%%%%%%%%%%
%%%%%%%%%%%%%%%%%%%%%%%%%%%%%%%%%%%%%%%%%%%%%%%%%%%%%%%%%%%%%%%%%%%%%%%%


%%%%%%%%%%%%%%%%%%%%%%%%%%%%%%%%%%%%%%%%%%%%%%%%%%%%%%%%%%%
\sect{Model}

A periodic banded logbox model. The band profile is flat ($\Theta(b)$).
The off diagonal elements are log-box:
%
\begin{align}
w_{nm} &= w_0 \eexp{-\epsilon} \\
\epsilon &\in \textrm{uniform  } [0,2\sigma]
\end{align}
%
The diagonal elements are set so that each row's sum will be zero.
%
\begin{align}
w_{nn} = -\sum_{m\ne n} w_{nm}  
\end{align}
%

\sect{PN plots}

\begin{figure}[H]
  \subfloat[Low sparsity]{
    \includegraphics{pta_low_s_nopin}
    
  }
  \subfloat[Sparse]{
    \includegraphics{pta_higher_s_nopin}
    
  }
  \\
  \subfloat[Very sparse]{
\includegraphics{pta_highest_s_nopin}
    
  }
\subfloat[Very sparse - logscale]{
\includegraphics{pta_highest_s_log_nopin}
    
  }
  \caption{{\bf PN} as a function of $\lambda$ for sparse systems.
  For all the plots $N=1000$ and $b=5$}
\end{figure}



%%%%%%%%%%%%%%%%%%%%%%%%%%%%%%%%%%%%%%%%%%%%%%%%%%%%%%%%%%%%%%%%%%%%
\sect{The effects of pinning}

By pinning we mean adding some elements to the diagonal,
that represent a spring attaching a ball to a fixed origin.


We used a uniform distribution on the range $[-0.3,0]$,
in order to discern the effect.


It seems that the only difference is some localization in
the low-eigenvalues. To highlight the low eigenvalues,
we have used log-scale for the $\lambda$ axis:

\begin{figure}[H]
\includegraphics{pta_low_s_log_nopin}
\includegraphics{pta_low_s_log_pin}
\caption{On the left, without pinning, and on the right with pinning}
\end{figure}


\end{document}
