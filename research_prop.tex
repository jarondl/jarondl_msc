%%%%%%%%%%%%%%%%%%%%%%%%%%%%%%%%%%%%%%%%%%%%%%%%%%%%%%%%%%%%%%%
%%%  notes
%%%%%%%%%%%%%%%%%%%%%%%%%%%%%%%%%%%%%%%%%%%%%%%%%%%%%%%%%%%%%%%

\documentclass[onecolumn,fleqn,notitlepage,secnumarabic]{revtex4}

% special 
\usepackage{ifthen}
\usepackage{ifpdf}
\usepackage{float}
\usepackage{color}

% fonts
\usepackage{latexsym}
\usepackage{amsmath} 
\usepackage{amssymb} 
\usepackage{bm}
\usepackage{wasysym}


\ifpdf
\usepackage{graphicx}
\usepackage{epstopdf}
\else
\usepackage{graphicx}
\usepackage{epsfig}
\fi

% packages added by jarondl
\usepackage{subfig}
\usepackage{verbatim} % for multiline comments
\usepackage{natbib} % change the bibliography style 
\usepackage{fancybox} % allows putting boxes with borders
\usepackage{cmap}  % for making pdf mathmode searchable
%\usepackage{sectsty}
\usepackage[pdftitle={Yaron de Leeuw's research proposal}]{hyperref}  % for hyperlinks in biblio. should be called last?

\graphicspath{{figures/},{PROG/figures/}}


%%%%%%%%%%%%%%%%%%%%%%%%%%%%%%%%%%%%%%%%%%%%%%%%%%%%%%%%%%%%%%%%

% NEW 
\newcommand{\abs}[1]{\left|#1\right|}
\newcommand{\varphiJ}{\bm{\varphi}}
\newcommand{\thetaJ}{\bm{\theta}}
%\renewcommand{\includegraphics}[2][0]{FIGURE}


% math symbols I
\newcommand{\sinc}{\mbox{sinc}}
\newcommand{\const}{\mbox{const}}
\newcommand{\trc}{\mbox{trace}}
\newcommand{\intt}{\int\!\!\!\!\int }
\newcommand{\ointt}{\int\!\!\!\!\int\!\!\!\!\!\circ\ }
\newcommand{\ar}{\mathsf r}
\newcommand{\im}{\mbox{Im}}
\newcommand{\re}{\mbox{Re}}

% math symbols II
\newcommand{\eexp}{\mbox{e}^}
\newcommand{\bra}{\left\langle}
\newcommand{\ket}{\right\rangle}

% Mass symbol
\newcommand{\mass}{\mathsf{m}} 
\newcommand{\rdisc}{\epsilon} 

% more math commands
\newcommand{\tbox}[1]{\mbox{\tiny #1}}
\newcommand{\bmsf}[1]{\bm{\mathsf{#1}}} 
\newcommand{\amatrix}[1]{\begin{matrix} #1 \end{matrix}} 
\newcommand{\pd}[2]{\frac{\partial #1}{\partial #2}}

% equations
\newcommand{\be}[1]{\begin{eqnarray}\ifthenelse{#1=-1}{\nonumber}{\ifthenelse{#1=0}{}{\label{e#1}}}}
\newcommand{\ee}{\end{eqnarray}} 

% arrangement
\newcommand{\hide}[1]{}
\newcommand{\drawline}{\begin{picture}(500,1)\line(1,0){500}\end{picture}}
\newcommand{\bitem}{$\bullet$ \ \ \ }
\newcommand{\Cn}[1]{\begin{center} #1 \end{center}}
\newcommand{\mpg}[2][1.0\hsize]{\begin{minipage}[b]{#1}{#2}\end{minipage}}
\newcommand{\mpgt}[2][1.0\hsize]{\begin{minipage}[t]{#1}{#2}\end{minipage}}



% extra math commands by jarondl
\newcommand{\inner}[2]{\left \langle #1 \middle| #2\right\rangle} % Inner product

%fminipage using fancybox package
\newenvironment{fminipage}%
  {\begin{Sbox}\begin{minipage}}%
  {\end{minipage}\end{Sbox}\fbox{\TheSbox}}


%%%%%%%%%%%%%%%%%%%%%%%%%%%%%%%%%%%%%%%%%%%%%%%%%%%%%%%%%%%%%%%%%%%%%%%%%%%

% Page setup
\setlength{\parindent}{0cm} 
\setlength{\parskip}{0.3cm} 

%%% Sections. The original revtex goes like this:
%\def\section{%
%  \@startsection
%    {section}%
%    {1}%
%    {\z@}%
%    {0.8cm \@plus1ex \@minus .2ex}%
%    {0.5cm}%
%    {\normalfont\small\bfseries}%
%}%
%\def{ \bf %
%  \@startsection
%    {subsection}%
%    {2}%
%    {\z@}%
%    {.8cm \@plus1ex \@minus .2ex}%
%    {.5cm}%
%    {\normalfont\small\bfseries}%
%}%
%%%%%%% And our version goes like this:
\makeatletter
\def\section{%
  \@startsection
    {section}%
    {1}%
    {\z@}%
    {0.8cm \@plus1ex \@minus .2ex}%
    {0.5cm}%
    {\Large\bf $=\!=\!=\!=\!=\!=\;$}%
}%
\def\subsection{%
  \@startsection
    {subsection}%
    {2}%
    {\z@}%
    {.8cm \@plus1ex \@minus .2ex}%
    {.5cm}%
    {\normalfont\small\bfseries$=\!=\!=\!=\;$}%
}%
%%%%%%%%%%  Here we deal with capitalization. The original revtex first, and then our version.
%\def\@hangfrom@section#1#2#3{\@hangfrom{#1#2}\MakeTextUppercase{#3}}%
%\def\@hangfroms@section#1#2{#1\MakeTextUppercase{#2}}%
\def\@hangfrom@section#1#2#3{\@hangfrom{#1#2}{#3}}%
\def\@hangfroms@section#1#2{#1{#2}}%
\makeatother


%%%%%%%%%%%%%%%%%%%%%%%%%%%%%%%%%%%%%%%%%%%%%%%%%%%%%%%%%%%%%
%%%%%%%%%%%%%%%%%%%%%%%%%%%%%%%%%%%%%%%%%%%%%%%%%%%%%%%%%%%%%
\begin{document}

\title{Diffusion properties of mesoscopic systems}

\author{Yaron de Leeuw \\ Adviser: Professor Doron Cohen}
\affiliation{Physics Department, Ben Gurion University of the Negev}
\date{\today}
\maketitle


%%%%%%%%%%%%%%%%%%%%%%%%%%%%%%%%%%%%%%%%%%%%%%%%%%%%%%%%%%%%%%%%%%%%%%%%%%%%%%%%%%%%%%%%%%%%
%%%%%%%%%%%%%%%%%%%%%%%%%%%%%%%%%%%%%%%%%%%%%%%%%%%%%%%%%%%%%%%%%%%%%%%%%%%%%%%%%%%%%%%%%%%
%%%%%%%%%%%%%%%%%%%%%%
\section{Work plan}

%%%%%%%%%%%

{ \bf Modeling .-- } $N$ interconnected sites constitue a network. A single particle is bound to the network. We denote by $p_n(t)$ the probability to find the particle on site $n$ at time $t$, so that $\sum_n p_n(t) =1$. The dynamics of the system are described by the rate equation:
\begin{align}
\frac{dp_n(t)}{dt} = \sum_m W_{nm}p_m(t)
\end{align}
Where $W_{nm}$ is the \emph{transition rate}, i.e.\ the rate at which probability moves from site $m$ to site $n$.
Because of probability conservation, we want to have $\sum_m W_{nm} = 0 \ \ \forall n$ , which we can achieve by setting $W_{nn} = -\sum_{m\ne n} W_{nm} $, meaning that for each site the sum of incoming transition rates negates the outgoing transitions.
The rate equation can also be written as a vectorial equation:$\boldsymbol{ \dot p } = \boldsymbol{W} \boldsymbol{p}$. In its basic form, the matrix $\boldsymbol{W}$ is symmetric (at the moment, see "Assymetric VRH" further down in this section), except for the main diagonal, which has values that ensure that each row's sum is zero.

%%%%%%%%%%%
The network (the values of $W_{nm}$) can be defined arbitrarily, but we wish to focus on networks that represent geometric systems, by defining the transition rates to depend on the distance betweeen randomly scattered points. One such network, with the rates $W_{nm}= e^{-r_{nm}/ \xi}$, where $r_{nm}$ is the distance between site $n$ and $m$, and $\xi$ is some scaling coefficient, was studied in \cite{Amir:2010:PRL}, and is of particular interest for us.

%%%%%%%%%%%
{ \bf dimensions.-- } The system may be in $1d$, $2d$, $3d$ etc., or in quasi-$1d$. Quasi-$1d$ relates either to $1d$ systems where there are bonds between sites beyond the nearest neighbors, or to banded $2d$ systems, where the length in one dimension is much smaller than in the other dimension. Both these systems have banded matrices, with bandwidth $b$.

%%%%%%%%%%%%%%%%%%%%%%%%%
{ \bf Survival probability - $\mathcal{P}(t)$.--} 
The survival probability is the probability to remain in the starting site. If the initial condition was $P_0(0)=1$, $P_i(0)=0 \textrm{  for  } i\neq 0$, then $\mathcal{P}(t)= P_0(t)$. The survival probability is directly related to the spectral properties of the transition matrices, being $\mathcal{P}(t) = \frac{1}{N}\sum_\lambda e^{\lambda t}$ where the $\lambda$s are the eigenvalues of the matrix.

%%%%%%%%%%%
{ \bf Diffusion.--}
Particles can diffuse through the system. In \cite{Alexander:1981:RMP} it is shown that in $1d$ the diffusion can found by finding the survival probability. CLARIFICATION NEEDED


THE QUESTIONS
%%%%%%%%%%%
{ \bf Geometrical implications.-- }  The statistics of the distances, and by extension  the statistics of the transition rates, reflects the geometric properties of the system. However, there is more information about the system than just its statistics. The question arises: Are the statistics all that is needed to understand the physics of the system? See the preliminary results for farther discussion \ref{sec:prelim}.

{ \bf Banded sparse matrices.-- } The conductance of quasi-$1d$ banded sparse matrices was studied numerically in \cite{Stotland:2010:PRB}. We wish to comprehend analytically what are the bandwidth implications in this model.

{ \bf Diffusion or Subdiffusion in $2d$.-- } The question of diffusion in most $1d$ systems was analytically solved in \cite{Alexander:1981:RMP}. Converting the distance statistics of scattered points to transition rate statistics, we find that subdiffusion is expected. The $2d$ case is, as for as we know, not yet analytically solved, and is much less clear. In \cite{Amir:2010:PRL} it is claimed that in low density systems subdiffusion of order $\sim log^d$ should occur. We wish to use the block-renormalization-group method to find out if this is indeed the case. We also wish to see if there is a transition between low densities and high densities.

{\bf Assymetric VRH.--} If the sites have different potentials, then the site occupation probabilites change. This can be acommodated by modifying the transition rates by Boltzmann's factor. This is called \emph{VRH- Variable Range Hopping}\cite{Ambegaokar:1971}. This is usually treated by modifying the transition rates symmetrically, but we know that the transition from higher to lower potential is more probable, and therefore the transition rates should not be symmetric. The question is what are the effects of this assymetry on the problem.


\section{Preliminary results} \label{sec:prelim}
{ \bf Geometrical implications.-- } In order to achieve insight to the geometric effects on the matrix spectrum, beyond the direct statistical influence, some numerical results were calculated. We have randomly scattered $200$ points on a $2d$ periodic surface. The transition matrix for this system was calculated, together with its eigenvalues. Now, the off-diagonal transition rates were randomly permuted, the new matrix was symmetrized, and its diagonal reset to keep the rows zero-summed. The cummulative eigenvalue distributions for the original and for the permuted matrices are plotted in figure \ref{fig:torus}. This is for a single realization.

\begin{figure}
    \includegraphics[clip, width=0.9\hsize]{torus}
    \caption{The geometric implications}\label{fig:torus}
\end{figure}


%\bibliographystyle{plainnat}
\bibliography{jarondl}
\end{document}
