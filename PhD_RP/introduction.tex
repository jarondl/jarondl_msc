\chapter{Introduction}


This proposal is about simple dynamical networks, with dynamics
that are described either by a stochastic rate equation
%
\begin{align}
  \frac{dp_n}{dt} \quad = \quad \sum_m w_{nm}p_m
\end{align}
%
or by the Schr\"{o}dinger equation:
%
\begin{align}
  \frac{d\psi_n}{dt} \quad = \quad \sum_m \mathcal{H}_{nm}\psi_m
\end{align}
%

Regardless of the governing equation, the focus is
on disordered sparse networks, with wide distribution of energies. 
Two related aspects of these networks will be studied: their energy spectrum,
and their spreading properties. More specifically, we wish to determine
whether the networks are diffusive ($S(t)\sim t$), or sub-diffusive, and
if they are indeed diffusive, what is the diffusion coefficient. 




To simplify the discussion, we concentrate on $d$-dimensional spatial networks,
in which each node has a defined location $x_n$, and each edge has an associated 
energy barrier $\epsilon_{nm}$. The transition rates are then defined to be:
%
\begin{align}
w_{nm} \quad = \quad w_0 \eexp{-\epsilon_{nm}}B(x_n-x_m)
\end{align}
%
Where $B(r)$ describes the dependence of the coupling on the
distance between sites.


The model we start with has its sites distributed randomly and uniformly throughout
a $d$-dimensional hypercube, with $B(r) = \exp(-r/\xi)$. The disorder is defined by the location
of the sites, and its effect on the connectivity of the network can be manipulated through the 
parameter $\xi$. 



%%%%%%%%%%%%%%%%%%%%%%%%%%%%%%%%%%%%
\section{Matrix categories}\label{sec:matrix_categories}

% Three categories:
% wnm = wmn  - no talk about steady state - equilibrium transient diffusion
%%            resnet
% wnm propto wmn - detailed balance - stochastic field
%           integral over it must be zero  boltzmann equil
%           should be added to the background
% wnm \ne wmn - assymetry - frustrated NESS

In general, the transition matrices have three symmetry categories:
\begin{itemize}
  \item 
    $w_{nm} = w_{mn}$ \ -\ In the symmetric case the system could reach
      equilibrium and ergodicity. The transient behavior might be diffusive. 
      This kind of system has the analog of a Resistor Network, which is often
      useful both analytically and numerically.
  \item
    $w_{nm} = \exp{-\frac{E_n-E_m}{T}} w_{mn}$\ -\ This is called detailed balance,
      and allows the system to reach Boltzmann's equilibrium. One could make
      the analogy of a "stochastic field", $\varepsilon(x) = \frac{E_n-E_m}{T}$.
      Any closed loop integral over such a field must equal zero.
      Note that any product of a symmetric operator with a diagonal operator
      can be said to satisfy detailed balance \cite{kolmogoroff_zur_1936}
  \item
    $w_{nm} \ne w_{mn}$\ -\ In the assymetric case, a Non Equilibrium Steady
      State (NESS) might be reached. 
\end{itemize}

%%%%%%%%%%%%%%%%%%%%%%%%%%%%%%%%%%%%%%%%%%%%%%%%%%%%%
%\section{TODO VRH}

%%%%%%%%%%%%%%%%%%%%%%%%%%%%%%%%%%%%%%%%%%%%
\section{Motivations for the model}
%% bullets - where does this model appear - balls and springs (leads to heat) - capacitors
%%             - conductance hopping

We have kept the model abstract and general on purpose, so that it
might fit a large number of physical situations. Some physical applications include:
%
\begin{itemize}
  \item
    Balls and Springs\ - \ A simple mechanical system of balls connected to each
      other by springs. The vibrational modes of the system are phonons, which conduct heat.
  \item
    Capacitors and Resistors\ -\ An electrical network of capacitors connected to each other
      by resistors.
  \item 
    Mott impurity hopping\ - \ This kind of conductance arises from hopping 
      between impurities.
\end{itemize}


%%%%%%%%%%%%%%%%%%%%%%%%%%%%%%%%%%%%%%%%%%%%%%%%%%%%%%%
\section{The Anderson localization}\label{sec:anderson}

In the Anderson model\cite{anderson_absence_1958}, a nearest neighbor network is formed
on a lattice. The bonds are all equal, and the on-site energies are random.
For $d=1$ and $d=2$, all states are localized, while for $d>2$, there exists
a transition between localized and extended modes. 


%The expressions for \emph{dc} conductivity and average conductance 
%\cite{kubo_statistical-mechanical_1957,greenwood,thouless_electrons_1974,braun_level_1997}
% can be written as
%
%\begin{align}
%\sigma &= \frac{\pi e^2}{m^2 V} \sum_{\alpha,\beta} 
%                 \left|\bra\alpha |\hat{p}|\beta\ket\right|^2 \delta(E_F-E_\alpha)\delta(E_F-E_\beta) \\
%\langle G \rangle &= \langle \sigma \rangle L^{d-2}
%\end{align}
%



There are several routines used to distinguish between exponentially localized and delocalized modes.
The most straight forward one seems to be the participation number (PN) 
\cite{edwards_numerical_1972}, which is defined by its inverse:
%
\begin{align}
PN^{-1} \ =\ \sum_i p_i^2
\end{align}
where $p_i$ is the probability to be in a specific site, equal to $|\psi_i|^2$ 
for a quantum wave function. For a fully extended mode $PN\sim N$, while for a state
occupying only one site $PN = 1 $.


Another idea called Thouless curvature or Thouless conductance 
\cite{edwards_numerical_1972,thouless_electrons_1974,braun_level_1997}, is based
on the idea that localized eigenmodes are almost not sensitive to boundary conditions. We apply a 
phase change $\phi$ across the boundary, (due to gauge invariance the specific spot is irrelevant),
and check the resulting change in the eigenvalues. 
In an ordered system with extended modes, $\phi=\pi$ will cause an energy
shift larger than the level spacing. 
Using perturbation theory on Kubo's equation \cite{kubo_statistical-mechanical_1957},
Thouless defined a conductance that is proportional to the diffusion: 
\begin{align}\label{eq:thouless}
\bra g_T\ket  \equiv \pi \frac{\bra \left|\left. \frac{\partial ^2 E^{(n)}}{\partial \phi^2}\right|_{\phi\rightarrow 0}\right|\ket}{\Delta}
\end{align}
where $\Delta$ is the level spacing.



Because we have defined the diagonal by requiring each row and column to nullify,
there always exists a special extended $\lambda=0$ mode with all probabilities equal. 
Actually, in our $d\le 2$ models, the localization length diverges as $\lambda\rightarrow 0$,
So that while the states are localized by definition, in practice they may be very wide
compared to the system length.



%%%%%%%%%%%%%%%%%%%%%%%%%%%%%%%%%%%%%%%%%%%%
\section{Modeling quantum diffusion with resistor networks}

%In past work \cite{cohen_wave_2000} 

In a model described by a Hamiltonian 
%
\begin{align}
\mathcal{H} = \textrm{diag}{E_n} + \epsilon {V_{nm}}
\end{align}
%
according to the first step of perturbation theory,
for very short time the transition probability is:
%
\begin{align}
 p(n, n_0)\ =\ \left| \epsilon \bra n| V |n_0\ket t\right|^2 \ = \ \epsilon^2 t^2\left| \bra n| V |n_0\ket \right|^2
\end{align}
%
This suggests that for some short time scale, the ballistic behavior
of the system might relate to the conductivity of the rate equation $G_{nm} = \left| \bra n| V |n_0\ket \right|^2$.




%%%%%%%%%%%%%%%%%%%%%%%%%%%%%%%%%%%%%%%%%%%%%%%
\section{Heat conductance}

One can take one of our suggested models, and couple each end to a different heat bath.
In steady state, it is expected that the heat current will obey Fourier's law:
%
\begin{align}
j = -\kappa\nabla T
\end{align}
%
Assuming $\kappa$ does not change significantly with temperature, a given 
temperature difference $\delta T$ over sample length $L$ should yield:
%
\begin{align}
j = \frac{\kappa \delta T}{L}
\end{align}
%
However, many one-dimensional models 
\cite{narayan_anomalous_2002,dhar_heat_2001,lepri_anomalous_1998,savin_heat_2002} 
exhibit sample size scaling of the form
%
\begin{align}
j         &\propto L^{\alpha-1}\quad [\alpha>0] \\
\kappa    &\propto L^\alpha
\end{align}

It is debated whether for $d=1$ systems
with momentum conservation,  $\alpha=1/3$,$\alpha=2/5$ or neither
\cite{narayan_anomalous_2002,delfini_comment_2008,dhar_dhar_2008,wang_power-law_2011}.
For several systems it has been shown that $\alpha$ depends on some tunable potential \cite{tong_wave_1999},
and it also depends on the spectrum of the baths \cite{dhar_heat_2001}.



The total heat conductance is defined by
%
\begin{align}
J = \int_0^\infty \rho(\lambda) g(\lambda) d\lambda
\end{align}
where $\rho(\omega)$ is the density of states, and $g(\lambda)$ is 
a transmission coefficient for a phonon at energy $\lambda$.
$g(\lambda)$ is highly affected by the localization length of this mode, 
and one way to find it is via Thouless' curvature \autoref{eq:thouless}.
The high frequency eigenmodes are strongly localized, hence only the
first part of the spectrum contributes to the heat flux. Based on this idea,
the heat flux can be estimated \cite{lepri_thermal_2001,bodyfelt_unpub},
by the number of wide modes.





 


%%%%%%%%%%%%%%%%%%%%%%%%%%%%%%%%%%%%%%%%%%%%%%%%%%%%%
\section{Banded matrices spectrum}


For wide bandwidth and uncorrelated matrix elements, the high eigenvalues
should follow the Wigner semicircle law ($g(\lambda) = \frac{2}{\pi R^2}\sqrt{R^2-\lambda^2}$)
\cite{erdos_local_2012,fyodorov_scaling_1991,wigner_characteristic_1955}. However,
the low eigenvalues can still follow other rules, allowing for a transition
between diffusion and subdiffusion.





%%%%%%%%%%%%%%%%%%%%%%%%%%%%%%%%%%%%%%%%%%%%%%%%%%%%%%
%\section{Sinai diffusion}




%For example, by adding spatial correlations, the behavior can be changed 
%from $S(t) \sim (\log t)^4$ to $S(t) \sim (\log t)^y$ with $y \in [4,\inf]$ \cite{stanley_generalisation_1987}.

%%%%%%%%%%%%%%%%%%%%%%%%%%%%%%%%%%%%%%%%%%%%%%%%%%%%%%%
%\section{Organization}

%The organization of this proposal is in parts.
%\autoref{part1} includes this introduction,
%the objectives of the proposal and further background. 
%In \autoref{part2} we present the work accomplished so far.
%In \autoref{part3} lies the appendix, including some derivations,
%and an article accepted for publication at PRE, available in the arXiv \cite{de_leeuw_diffusion_2012}.
