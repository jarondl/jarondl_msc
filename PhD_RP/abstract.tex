\chapter*{Abstract}


We study transport and spreading in sparse disordered networks. The networks are 
constructed by assigning spatial locations to sites. Their dynamics are
described by a rate equation, with transition rates
defined both though the distance between sites as well as by a
coupling parameter. Thus, by randomizing either the locations of the sites
or the coupling parameter, we can induce disorder.



By allowing the network elements to differ by orders of magnitude,
with most of the elements vanishingly small, we create a "sparse" network.
Our interest lies in the long term transport properties of these sparse systems. 


We approach the subject of transport by looking at the spectral properties
of these networks, as well as by their resistor network analogies. We see 
that in pure $1d$ this sparsity causes subdiffusion, while in quasi $1d$ and
higher dimensions the system remains diffusive but with a reduced diffusion coefficient. 


The \emph{ERH - Effective Range Hopping -} method is suggested to estimate this diffusion 
coefficient by combining ideas from the well established \emph{VRH} scheme with 
ideas from percolation theory.


The numerical results for the diffusion coefficient are compared to the linear
estimate and to the improved \emph{ERH} result.
