\chapter*{Abstract}


We study transport and spreading in sparse disordered networks. 
The dynamics of the networks are defined by rate equations or Hamiltonians,
either way defined by real matrices. 
By allowing matrix elements to differ by orders of magnitude,
with most of the elements vanishingly small, we create a ``sparse'' network.
Our interest lies in the transport properties of these sparse systems. 



Our main interest is to tie together several approaches 
used to study transport, in particular the resistor network 
analogy, the survival probability spectral derivation, and the
heat transport covariance matrix formalism. There are reasons to believe
that these are all basically intertwined facets of one idea, and each
method could be better understood by knowing this relation.


For stochastic systems we see 
that in pure $1d$ this sparsity causes subdiffusion, while in quasi $1d$ and
higher dimensions the system remains diffusive but with a reduced diffusion coefficient. 
The \emph{ERH - Effective Range Hopping -} method is suggested to estimate this diffusion 
coefficient by combining ideas from the well established \emph{VRH} scheme with 
ideas from percolation theory. The numerical results for the diffusion coefficient are compared to the linear
estimate and to the improved \emph{ERH} result.





Further generalizations are discussed, including multiple interacting particles,
extending the \emph{ERH} result to quantum spreading and
asymmetric matrices. Another proposed model is a $2d$ variant of 
the Sinai model.
