\chapter{Background}

 
% Also, some of the MSc_RP intro.

% Analogy to resistor network and to springs and masses.


\section{Transition matrices}

Our system is a $d$-dimensional network, and it is
described by its nodes and edges. The nodes represent 
either {\em sites} in real space or {\em levels} in energy space, 
and our particle (or particles, see \ref{sec:discreteness}) may
occupy them. The edges of the network represent transition rates 
between the nodes. The dynamics of the model are described by 
a rate equation:
\[
\frac{dp_n}{dt} \quad = \quad \sum_m w_{nm}p_m
\]
Where the transition matrix 
\[
\mathbf{W} \quad =  \quad \left[ w_{nm}\right]_{N\times N}
\]
Is the crux of this work.

Because of conservation of probability, we set the diagonal
elements to
\[ w_{nn}\quad = \quad -\sum_{m(\ne n)} w_{mn}\]


%%%%%%%%%%%%%%%%%%%%%%%%%%%%%%
\section{Definition of the model}


The proposal regards a spatial network where each node
has a defined location $x_n$. The transition rates $w_{nm}$
are constructed by the expression
%
\beq
w_{nm} \quad = \quad w_0 \eexp{-\epsilon_{nm}}B(x_n-x_m)
\eeq
%
where $B(r)$ describes the dependence of the coupling on the
distance between sites, and $\epsilon$ represent activation
energies required for a transition.


The $b(r)$ matrix is a function a euclidean distance matrix,
a class a matrices that is widely researched ..... TODO

%%%%%%%%%%
\subsection{The random site hopping model}




For presentation purpose we regard the nodes of the network as {\em sites}, 
each having a location~$x_n$. In particular (but not exclusively) we are interested 
in a model where the rates depend exponentially on the distance 
between randomly distributed sites, namely $w_{nm}\propto \exp(|x_n-x_m|/\xi)$. 
One can characterize such a system by a sparsity parameter~$s$ 
that reflects the connectivity of the network. For a random site model
the natural definition is $s=\xi/r_0$, where $r_0$ is the average distance 
between neighboring sites. 



The models that we address are related and motivated  
by various physical problems, for example: 
phonon propagation in disordered solids \cite{phn1,phn2,amir}; 
Mott hopping conductance \cite{mott,miller,AHL,Halp,pollak,VRHbook};
transport in oil reservoirs \cite{aa1,aa2};
conductance of ballistic rings \cite{kbd};
and energy absorption by trapped atoms \cite{kbw}. 
%
Optionally these models can be fabricated by combining oscillators: 
say mechanical springs or electrical RC elements. 

%%%%%%%%%%%%%%%%%%
\section{The study of networks}
%%
%   
In all these examples the issue is to understand how 
the {\em transport} is affected by the {\em sparsity} of a network.  
If the rates are induced by a driving source, this issue can be phrased as  
going {\em beyond} the familiar framework of Linear Response Theory (LRT), 
as explained below.  

%%%%%%%%%%%%%%%%%%%
\section{Diffusion} 
Our interest is focused on the diffusion coefficient $D$ that characterizes the 
long time dynamics of a spreading distribution. It can be defined or deduced 
either from the variance ${S(t) \sim Dt}$ or from the decay of the 
survival probability ${\mathcal{P}(t) \sim (D t)^{-d/2}}$. Hence it is 
related to the spectral properties of the transition rate matrix 
%
\beq
\bm{w} \ \ = \ \ \{w_{nm}\}
\eeq
%
Exploiting the formal analogy with a resistor network calculation \cite{miller},  
namely $w_{nm}$ are like connectors and $D$ is like conductivity, 
one realizes that $D$ is given by a semi-linear functional $D[\bm{w}]$ 
that has the property ${D[\lambda \bm{w}] = \lambda D[\bm{w}]}$, 
while in general ${D[\bm{w}^a+\bm{w}^b] > D[\bm{w}^a]+D[\bm{w}^b]}$ instead of equality.      


%%%%%%%%%%%%%%%%%%%
\section{Subdiffusion}
In the 1D ($d{=}1$) case, it is well known \cite{alexander} that $D$ can display an abrupt 
percolation-like transition from diffusive (${D>0}$) to sub-diffusive (${D=0}$) 
behavior, as the sparsity parameter drops below the critical value ${s_c=1}$.
Similar anomalies are found for fractal structures with ${d<2}$, see \cite{granek,havlin}. 
A question arises whether such a transition might happen in higher dimensions.  
In \cite{amir} the spectral properties in the 2D ($d{=}2$) case 
have been investigated: on the basis of the renormalization group (RG) procedure 
it has been deduced that $\mathcal{P}(t)$ decays in a logarithmic way, 
indicating anomalous (sub) diffusion.  


%%%%%%%%%%%%%%%%%%%%%
\section{Variable range hopping}
It should be clear that there are two major routes in developing  
a theory for~$D$. Instead of deducing it from spectral properties 
as in \cite{amir}, one can try to find ways to evaluate it directly 
via a resistor network calculation \cite{miller,AHL,Halp,pollak,VRHbook}, 
leading in the standard Mott problem to the Variable Range Hopping (VRH)
estimate for~$D$.   
%
In \cite{kbd,kbw,slk} this approach has been extended 
to handle ``sparse" banded matrices whose elements have log-wide distribution, 
leading to a generalized VRH estimate. 
In what follows we pursue the same direction and obtain an 
improved estimate for~$D$ that we call Effective Range Hopping (ERH).
Using this approach we show that in the 2D case, as $s$ becomes small, 
the functional $D[\bm{w}]$ exhibits a smooth crossover from ``linear" behavior  
to ``semi-linear" VRH-type dependence.  


%%%%%%%%%%%%%%%%%%%%%
\section{Sparsity vs percolation}
The problem that we consider is a variant of the percolation problem \cite{aa1,aa2}:
Instead of considering a bi-modal distribution (``zeros" and ``ones")
we consider a log-wide distribution of rates 
\cite{halperin_remarks_1989}, 
for which the median is much smaller than the mean value. 
We call such network ``sparse" (with quotation marks) because 
the large elements constitute a minority.

%%%%%%%%%%%%%%%%%%%%%%
\section{Anderson localization}
Disregarding the ``sparsity" issue, the model that we are considering 
is a close relative of the Anderson localization problem.
In the problem that we discuss here all the off diagonal elements 
are positive, while the negative diagonal elements compensate them.
%
Since all the  off diagonal elements are positive numbers, 
it is clear that we cannot have ``destructive interference", 
and therefore we do not have genuine Anderson localization. 
Therefore in general we might have diffusion, even in 1D. 
In 2D we have a percolation threshold, which is again 
not like Anderson localization. For further discussion see 
for example the discussion of fractons in \cite{havlin_diffusion_1987}.


%%%%%%%%%%%%%%%%%%%%%
\section{Debye law}
In the standard Anderson model the eigenvalues form 
a band ${\lambda \in [-\lambda_c,\lambda_c]}$.
The states at the edge of the band are always localized.
The states in the middle of the band might be de-localized if ${d>2}$.  
%
The localization in a disordered elastic medium had been 
studied \cite{loc}. Disregarding the ``sparsity",  
it is the same problem that we are considering here.
It has been found that the spectrum is ${\lambda \in [0,\lambda_c]}$.
The ground state is always the ${\lambda=0}$ uniform state.
The localization length diverges in the limit ${\lambda \rightarrow 0}$.
Consequently the Debye density of states is not violated:
the spectrum is asymptotically the same as that of a 
diffusive (non-disordered) lattice. It follows that the 
survival probability should be like that of diffusive 
system, and therefore we also expect diffusive behavior
for the transport: spreading that obeys a diffusion equation.


%%%%%%%%%
\section{Analogy to an electrical circuit}

We may think of a "ladder" of capacitors and resistors as portrayed in the diagram.



%%%%%%%%%
\section{Analogy to a masses and springs mesh} 

The same network may describe equal masses (which 
we can simply define as $1$) distributed in space,
connected by springs. The locations of the masses will
obey a second degree differential equation:

\rmrk{correct this!}
%
%\beq
%\frac{d^2x_n}{dt^2} \quad = \quad -\sum_m\frac{k_{nm}} (x_n)
%\eeq
%
Where $k_{nm}$ is the spring 


% based on the msc rp
\section{msc}

There are diverse examples for extended systems whose dynamics is described by a rate equation. This include in particular the analysis of "random walk" where the transitions are between sites of a network. Another example is the study of energy absorption due to Fermi-Golden-Rule transitions between energy levels. In both cases there are two common questions that arise: (1) What is the survival probability of a particle  that has been prepared in a given site; (2) Is there normal diffusion or maybe only sub-diffusion, and how it is related to the survival probability.

In recent years there is a growing interest in systems where the transition rates have log-wide distribution. This means that the values of the rates
are distributed over many decades as in the case of log-normal or log-box distributions. Such "glassy" or "sparse" systems can be regarded as a "resistor network", and the analysis might be inspired by percolation theory, variable range hopping phenomenology, and the renormalization-group methods.

While the theory of 1D networks with near-neighbour transitions is quite complete, the more general case of quasi-1D / 2D and possibly higher dimensions lacks a unifying framework, and there are numerous open questions that we would like to study as outlined below.

%%%%%%%%%%%%%%%%%%%%
{ \bf Modeling .-- } $N$ interconnected sites constitue a network. A single particle is bound to the network. We denote by $p_n(t)$ the probability to find the particle on site $n$ at time $t$, so that $\sum_n p_n(t) \;=\;1$. The dynamics of the system are described by the rate equation:
\begin{align}
\frac{dp_n(t)}{dt} \;=\; \sum_m W_{nm}p_m(t)
\end{align}
Where $W_{nm}$ is the \emph{transition rate}, i.e.\ the rate at which probability moves from site $m$ to site $n$.
Because of probability conservation, we want to have $\sum_m W_{nm} \;=\; 0 \ \ \forall n$ , which we can achieve by setting $W_{nn} \;=\; -\sum_{m\ne n} W_{nm} $, meaning that for each site the sum of incoming transition rates negates the outgoing transitions.
The rate equation can also be written as a vectorial equation:$\boldsymbol{ \dot p } \;=\; \boldsymbol{W} \boldsymbol{p}$. In its basic form, the matrix $\boldsymbol{W}$ is symmetric (at the moment, see "Assymetric VRH" further down in this section), except for the main diagonal, which has values that ensure that each row's sum is zero.

%%%%%%%%%%%
The network (the values of $W_{nm}$) can be defined arbitrarily, but we wish to focus on networks that represent geometric systems, by defining the transition rates to depend on the distance betweeen randomly scattered points\cite{Mezard:1999:NPB}. One such network, with the rates defined as:
\begin{align} \label{eq:exp_rates}
  W_{nm}\;=\; w_0 e^{(r_0-r_{nm})/ \xi}
\end{align}
Where $r_{nm}$ is the distance between site $n$ and $m$, $r_0$ is the typical distance between points, $w_0$ is the transition rate between points at distance $r_0$, and $\xi$ is a scaling coefficient. In general, we are going to define $r_0 = n^{-1/d}$, where $n=\frac{N}{V}$ is the site denisty, and we are going to rescale time by setting $w_0=1$. This model was studied in \cite{Amir:2010:PRL}, and is of particular interest for us.

%%%%%%%%%%%
{ \bf Dimensions.-- } The system may be in $1d$, $2d$, $3d$ etc., or in quasi-$1d$. The $1d$ system has been studied, among others, in\cite{Parris:1986}\cite{Alexander:1981:RMP}\cite{AslangulChvosta:1995}.  Quasi-$1d$ relates either to $1d$ systems where there are bonds between sites beyond the nearest neighbors, or to $2d$ systems with finite width (strip). Both these systems have banded matrices, with bandwidth $b$.

%%%%%%%%%%%%%%%%%%%%%%%%%
{ \bf Survival probability - $\mathcal{P}(t)$.--} 
The survival probability is the probability to remain in the starting site. If the initial condition was $p_0(0)\;=\;1$, $p_i(0)\;=\;0 \textrm{  for  } i\neq 0$, then $\mathcal{P}(t)\;=\; p_0(t)$. The survival probability is directly related to the spectral properties of the transition matrices, and it can be shown that 
\begin{align} \label{eq:p_t_spectrum}
\mathcal{P}(t) \;=\; \frac{1}{N}\sum_\lambda e^{\lambda t} \;\rightarrow\;\frac{1}{N}\int e^{\lambda t}g(\lambda)d\lambda
\end{align}
where the $\lambda$s are the eigenvalues of the matrix, namely that the survival probability is the Laplace transform of the eigenvalue density.

%%%%%%%%%%%
{ \bf Transport and Spreading.--}  %add some S(t) or g, and add \sqrt{S(t)}\mathcal{P}(t)
A particle can be transmitted through the system from one end to the other. This transport can be characterized in different ways. One way is to calculate the spreading $S(t)$, which is the variance (second moment) of the particle location, i.e:
\begin{align}
  S(t) \;=\; \sum_n \left(r_n(t)\right)^2 p_n  %-\overline{r}(t)
\end{align}
Where $r_n$ is the location of the $n$th site. The survival probability is related to the diffusion because of scaling considerations by:
% In \cite{Alexander:1981:RMP} it is shown that in $1d$ the spreading is related to the survival probability by
\begin{align}
\mathcal{P}(t) &\;=\; \left(2\pi \frac{S(t)}{r_0^2}\right)^{-d/2}
\end{align}
By definition, diffusive systems obey: 
\begin{align}
S(t) &\;=\; 2Dt  \\
\mathcal{P}(t) &\;=\; \left(2\pi \frac{S(t)}{r_0^2}\right)^{-d/2} \;=\; \left(4\pi \frac{Dt}{r_0^2} \right)^{-d/2}
\end{align}
We can combine this result with \ref{eq:p_t_spectrum} to obtain a relation between $g(\lambda)$ and D:
\begin{align}
    g(\lambda) &\;=\; \mathcal{L}^{-1}[\mathcal{P}(t)] \;=\; \mathcal{L}^{-1}\left[ \left(4\pi \frac{Dt}{r_0^2} \right)^{-d/2}\right] \;=\; \frac{\lambda^{\frac{d}{2}-1}}{\Gamma\left(\frac{d}{2}\right)\left(4\pi \frac{D}{r_0^2}\right)^{d/2}}  \\
    C(\lambda) &\;=\; \int_{\infty}^{\lambda} g(\lambda')d\lambda' \;=\; \frac{\lambda^{\frac{d}{2}}}{\frac{d}{2}\Gamma\left(\frac{d}{2}\right)\left(4\pi \frac{D}{r_0^2}\right)^{d/2}}
\label{eq:C_D}
\end{align}


%%%%%%%%%
\section{Spreading of wavepackets in the quantum case}

%%%%%%%%%
\section{Localization}




