%%%%%%%%%%%%%%%%%%%%%%%%%%%%%%%%%%%%%%%%%%%%%%%%%%%%%%%%%%%%%%%
%%%  notes
%%%%%%%%%%%%%%%%%%%%%%%%%%%%%%%%%%%%%%%%%%%%%%%%%%%%%%%%%%%%%%%

\documentclass[onecolumn,fleqn,notitlepage,secnumarabic]{revtex4}

% special 
\usepackage{ifthen}
\usepackage{ifpdf}
\usepackage{float}
\usepackage{color}

% fonts
\usepackage{latexsym}
\usepackage{amsmath} 
\usepackage{amssymb} 
\usepackage{bm}
\usepackage{wasysym}


\ifpdf
\usepackage{graphicx}
\usepackage{epstopdf}
\else
\usepackage{graphicx}
\usepackage{epsfig}
\fi

% packages added by jarondl
%\usepackage{subfig}  %% causes odd captions
\usepackage{verbatim} % for multiline comments
\usepackage{natbib} % change the bibliography style 
\usepackage{fancybox} % allows putting boxes with borders
\usepackage{cmap}  % for making pdf mathmode searchable
%\usepackage{sectsty}
\usepackage[pdftitle={Yaron de Leeuw's research proposal}]{hyperref}  % for hyperlinks in biblio. should be called last?

\graphicspath{{figures_rp/},{PROG/figures/}}


%%%%%%%%%%%%%%%%%%%%%%%%%%%%%%%%%%%%%%%%%%%%%%%%%%%%%%%%%%%%%%%%

% NEW 
\newcommand{\abs}[1]{\left|#1\right|}
\newcommand{\varphiJ}{\bm{\varphi}}
\newcommand{\thetaJ}{\bm{\theta}}
%\renewcommand{\includegraphics}[2][0]{FIGURE}


% math symbols I
\newcommand{\sinc}{\mbox{sinc}}
\newcommand{\const}{\mbox{const}}
\newcommand{\trc}{\mbox{trace}}
\newcommand{\intt}{\int\!\!\!\!\int }
\newcommand{\ointt}{\int\!\!\!\!\int\!\!\!\!\!\circ\ }
\newcommand{\ar}{\mathsf r}
\newcommand{\im}{\mbox{Im}}
\newcommand{\re}{\mbox{Re}}

% math symbols II
\newcommand{\eexp}{\mbox{e}^}
\newcommand{\bra}{\left\langle}
\newcommand{\ket}{\right\rangle}

% Mass symbol
\newcommand{\mass}{\mathsf{m}} 
\newcommand{\rdisc}{\epsilon} 

% more math commands
\newcommand{\tbox}[1]{\mbox{\tiny #1}}
\newcommand{\bmsf}[1]{\bm{\mathsf{#1}}} 
\newcommand{\amatrix}[1]{\begin{matrix} #1 \end{matrix}} 
\newcommand{\pd}[2]{\frac{\partial #1}{\partial #2}}

% equations
\newcommand{\be}[1]{\begin{eqnarray}\ifthenelse{#1=-1}{\nonumber}{\ifthenelse{#1=0}{}{\label{e#1}}}}
\newcommand{\ee}{\end{eqnarray}} 

% arrangement
\newcommand{\hide}[1]{}
\newcommand{\drawline}{\begin{picture}(500,1)\line(1,0){500}\end{picture}}
\newcommand{\bitem}{$\bullet$ \ \ \ }
\newcommand{\Cn}[1]{\begin{center} #1 \end{center}}
\newcommand{\mpg}[2][1.0\hsize]{\begin{minipage}[b]{#1}{#2}\end{minipage}}
\newcommand{\mpgt}[2][1.0\hsize]{\begin{minipage}[t]{#1}{#2}\end{minipage}}



% extra math commands by jarondl
\newcommand{\inner}[2]{\left \langle #1 \middle| #2\right\rangle} % Inner product

%fminipage using fancybox package
\newenvironment{fminipage}%
  {\begin{Sbox}\begin{minipage}}%
  {\end{minipage}\end{Sbox}\fbox{\TheSbox}}


%%%%%%%%%%%%%%%%%%%%%%%%%%%%%%%%%%%%%%%%%%%%%%%%%%%%%%%%%%%%%%%%%%%%%%%%%%%

% Page setup
\setlength{\parindent}{0cm} 
\setlength{\parskip}{0.3cm} 

%%% Sections. The original revtex goes like this:
%\def\section{%
%  \@startsection
%    {section}%
%    {1}%
%    {\z@}%
%    {0.8cm \@plus1ex \@minus .2ex}%
%    {0.5cm}%
%    {\normalfont\small\bfseries}%
%}%
%\def{ \bf %
%  \@startsection
%    {subsection}%
%    {2}%
%    {\z@}%
%    {.8cm \@plus1ex \@minus .2ex}%
%    {.5cm}%
%    {\normalfont\small\bfseries}%
%}%
%%%%%%% And our version goes like this:
\makeatletter
\def\section{%
  \@startsection
    {section}%
    {1}%
    {\z@}%
    {0.8cm \@plus1ex \@minus .2ex}%
    {0.5cm}%
    {\Large\bf $=\!=\!=\!=\!=\!=\;$}%
}%
\def\subsection{%
  \@startsection
    {subsection}%
    {2}%
    {\z@}%
    {.8cm \@plus1ex \@minus .2ex}%
    {.5cm}%
    {\normalfont\small\bfseries$=\!=\!=\!=\;$}%
}%
%%%%%%%%%%  Here we deal with capitalization. The original revtex first, and then our version.
%\def\@hangfrom@section#1#2#3{\@hangfrom{#1#2}\MakeTextUppercase{#3}}%
%\def\@hangfroms@section#1#2{#1\MakeTextUppercase{#2}}%
\def\@hangfrom@section#1#2#3{\@hangfrom{#1#2}{#3}}%
\def\@hangfroms@section#1#2{#1{#2}}%
\makeatother


%%%%%%%%%%%%%%%%%%%%%%%%%%%%%%%%%%%%%%%%%%%%%%%%%%%%%%%%%%%%%
%%%%%%%%%%%%%%%%%%%%%%%%%%%%%%%%%%%%%%%%%%%%%%%%%%%%%%%%%%%%%
\begin{document}

\title{Diffusion properties of mesoscopic systems}

\author{Yaron de Leeuw \\ Adviser: Professor Doron Cohen}
\affiliation{Physics Department, Ben Gurion University of the Negev}
\date{\today}
\maketitle


%%%%%%%%%%%%%%%%%%%%%%%%%%%%%%%%%%%%%%%%%%%%%%%%%%%%%%%%%%%%%%%%%%%%%%%%%%%%%%%%%%%%%%%%%%%%
%%%%%%%%%%%%%%%%%%%%%%%%%%%%%%%%%%%%%%%%%%%%%%%%%%%%%%%%%%%%%%%%%%%%%%%%%%%%%%%%%%%%%%%%%%%
%%%%%%%%%%%%%%%%%%%%%%
\section{Background}

%%%%%%%%%%%

{ \bf Modeling .-- } $N$ interconnected sites constitue a network. A single particle is bound to the network. We denote by $p_n(t)$ the probability to find the particle on site $n$ at time $t$, so that $\sum_n p_n(t) \;=\;1$. The dynamics of the system are described by the rate equation:
\begin{align}
\frac{dp_n(t)}{dt} \;=\; \sum_m W_{nm}p_m(t)
\end{align}
Where $W_{nm}$ is the \emph{transition rate}, i.e.\ the rate at which probability moves from site $m$ to site $n$.
Because of probability conservation, we want to have $\sum_m W_{nm} \;=\; 0 \ \ \forall n$ , which we can achieve by setting $W_{nn} \;=\; -\sum_{m\ne n} W_{nm} $, meaning that for each site the sum of incoming transition rates negates the outgoing transitions.
The rate equation can also be written as a vectorial equation:$\boldsymbol{ \dot p } \;=\; \boldsymbol{W} \boldsymbol{p}$. In its basic form, the matrix $\boldsymbol{W}$ is symmetric (at the moment, see "Assymetric VRH" further down in this section), except for the main diagonal, which has values that ensure that each row's sum is zero.

Quantum spreading and transport is another subject we wish to tackle, following the lead of \cite{Jayannavar:1991}\cite{Guarneri:1989}\cite{Guarneri:1993}.

%%%%%%%%%%%
The network (the values of $W_{nm}$) can be defined arbitrarily, but we wish to focus on networks that represent geometric systems, by defining the transition rates to depend on the distance betweeen randomly scattered points\cite{Mezard:1999:NPB}. One such network, with the rates defined as:
\begin{align} \label{eq:exp_rates}
  W_{nm}\;=\; w_0 e^{(r_0-r_{nm})/ \xi}
\end{align}
Where $r_{nm}$ is the distance between site $n$ and $m$, $r_0$ is the typical distance between points, $w_0$ is the transition rate between points at distance $r_0$, and $\xi$ is a scaling coefficient, was studied in \cite{Amir:2010:PRL}, and is of particular interest for us.

%%%%%%%%%%%
{ \bf Dimensions.-- } The system may be in $1d$, $2d$, $3d$ etc., or in quasi-$1d$. The $1d$ system has been studied, among others, in\cite{Parris:1986}\cite{Alexander:1981:RMP}\cite{AslangulChvosta:1995}.  Quasi-$1d$ relates either to $1d$ systems where there are bonds between sites beyond the nearest neighbors, or to $2d$ systems with finite width (strip). Both these systems have banded matrices, with bandwidth $b$.

%%%%%%%%%%%%%%%%%%%%%%%%%
{ \bf Survival probability - $\mathcal{P}(t)$.--} 
The survival probability is the probability to remain in the starting site. If the initial condition was $p_0(0)\;=\;1$, $p_i(0)\;=\;0 \textrm{  for  } i\neq 0$, then $\mathcal{P}(t)\;=\; p_0(t)$. The survival probability is directly related to the spectral properties of the transition matrices, and it can be shown that 
\begin{align} \label{eq:p_t_spectrum}
\mathcal{P}(t) \;=\; \frac{1}{N}\sum_\lambda e^{\lambda t} \;\rightarrow\;\frac{1}{N}\int e^{\lambda t}g(\lambda)d\lambda
\end{align}
where the $\lambda$s are the eigenvalues of the matrix, namely that the survival probability is the Laplace transform of the eigenvalue density.

%%%%%%%%%%%
{ \bf Transport and Spreading.--}  %add some S(t) or g, and add \sqrt{S(t)}\mathcal{P}(t)
A particle can be transmitted through the system from one end to the other. This transport can be characterized in different ways. One way is to calculate the spreading $S(t)$, which is the Root-Mean-Square of the particle location, i.e:
\begin{align}
  S(t) \;=\; \sum_n (r_n(t)  )^2 p_n  %-\overline{r}(t)
\end{align}
Where $r_n$ is the location of the $n$th site. In \cite{Alexander:1981:RMP} it is shown that in $1d$ the spreading is related to the survival probability by
\begin{align}
S(t) \;=\; \frac{1}{2\pi\mathcal{P}^2(t)}
\end{align}
For diffusive systems, 
\begin{align}
S(t) &\;=\; 2Dt  \\
\mathcal{P}(t) &\;=\; \sqrt{\frac{1}{2\pi S(t)}} \;=\; \sqrt{\frac{1}{4\pi D t}}
\end{align}
We can combine this result with \eqref{eq:p_t_spectrum} to obtain a relation between $g(\lambda)$ and D:
\begin{align}
    g(\lambda) &\;=\; \mathcal{L}^{-1}[\mathcal{P}(t)] \;=\; \mathcal{L}^{-1}\left[\frac{1}{\sqrt{4\pi D t}}\right] \;=\; \frac{1}{\sqrt{4\pi^2 D\lambda}} \\
    C(\lambda) &\;=\; \int_{\infty}^{\lambda} g(\lambda')d\lambda' \;=\; \sqrt{\frac{\lambda}{\pi^2 D}} \label{eq:C_D}
\end{align}

We do not know yet if a similar relation holds in $2d$. Another approach is the resistor network one. If we think of the network as a network of resistors, then we can characterize the transport by the conductance between two points. 

%%%%%%%%%%%%%%%%%%%
\section{Work plan}


%%%%%%%%%%%
{ \bf Geometrical implications.-- } The geometric properties of the system are reflected in the statistics of the distances, and by extension in the statistics of the transition rates. However, there is more information in the $W_{nm}$ matrix than just its statistics. The question arises: Are the statistics all that is needed to understand the physics of the system? See the preliminary results for farther discussion \ref{sec:prelim}.

{ \bf Banded sparse matrices.-- } The conductance of quasi-$1d$ banded sparse matrices was studied numerically in \cite{Stotland:2010:PRB}. There, they use \emph{Variable-Range-Hopping} in the sparse regime, where 
$(\text{sparsity}\cdot \text{bandwidth}) \ll 1$. However, in this work it is not clear what are the limits on either sparsity or bandwidth, and in particular where is the cross between the validity regime of VRH and that of SLRT. We will try to understand this issue analytically.

{ \bf Diffusion or Subdiffusion in $2d$.-- } The question of diffusion in most $1d$ systems was analytically solved in \cite{Alexander:1981:RMP}. The $2d$ case is, as for as we know, not yet analytically solved, and is much less clear. In \cite{Amir:2010:PRL} it is claimed that in low density systems subdiffusion of order $\sim log^d$ should occur. We wish to use the block-renormalization-group method to find out if this is indeed the case. We also wish to see if there is a transition between low densities and high densities.

{\bf Assymetric VRH.--} If the sites have different potentials, then the site occupation probabilites change. This can be acommodated for by modifying the transition rates by Boltzmann's factor. This is called \emph{VRH- Variable Range Hopping}\cite{Ambegaokar:1971}, and has been widely studied. The common practice is to treat the system as a symmetric resistor network, but we want to ask if there are cases where this reduction to a symmetrical network is not valid.

%%%%%%%%%%%%%%%%%%%%%%%%
\section{Preliminary results} \label{sec:prelim}

{ \bf Geometrical implications.-- } In order to achieve insight to the geometric effects on the matrix spectrum, beyond the direct statistical influence, some numerical results were calculated. We have randomly scattered $200$ points on a $2d$ periodic surface. The transition rates between the sites were calculated as usual. Another network was created, using the same rates, but in a randomly permuted order. That way, the statistics of the rates were preserved, but the rest of the data not. The cummulative eigenvalue distributions for the original and for the permuted matrices are plotted in figure \ref{fig:torus}. This is for a single realization, but the distinct difference between the spectra is present in all of the realizations that we have checked.

\begin{figure}
    \includegraphics[clip, width=0.9\hsize]{torus}
    \caption{{ \bf The geometric implications -} $N\;=\;800$ points were randomly scattered on a $2d$ $(1\times 1)$ surface. The transmission rates were calculated according to \eqref{eq:exp_rates}, with $\frac{\xi}{r_0} \;=\; 30$, and $w_0\;=\;1$. Then, the transition rates were shuffled, and the plot redrawn.}
    \label{fig:torus}
\end{figure}


%%%%%%%%%%%%%%

\begin{figure}
    \includegraphics[clip, width=0.9\hsize]{quasi_1d}
    \caption{Quasi-1d. No fitting parameters were used. $N\;=\;800$ points were randomly scattered on a $1d$ line. The transition rates between them are according to \eqref{eq:exp_dep}, with $\epsilon=\frac{\xi}{r_0}=10$. Each system was truncated at a different bandwidth, as stated in the legend. The dashed line marks \eqref{eq:C_D}, with $D$ calculated using the resistor network method \eqref{eq:resnet}. }
    \label{fig:quasi_1d}
\end{figure}

%%%%%%%%%%%%
{ \bf The exponential dependence model in various dimensions.--} In this model we define the rates as 
\begin{align} \label{eq:exp_dep}
  W_{nm}\;=\; w_0 e^{(r_0-r_{nm})/ \xi}
\end{align}
Where $r_0$ is the typical distance between points, $w_0$ is the transition rate between points at distance $r_0$, and $\xi$ is a scaling coefficient. It can be shown that the distances of randomly scattered points have a distribution described by 
\begin{align}
  f(r)dr \;=\; e^{-\frac{\Omega_d}{2d} \left(\frac{r}{r_0}\right)^d} \frac{\Omega_d}{2} \left(\frac{r}{r_0}\right)^{d-1}\frac{dr}{r_0}
\end{align}
where $n$ is the density, $d$ the dimension, and $\Omega_d$ is the $d$ dimensional solid angle. The distribution of the rates can be found by replacing the random variable ($|f_w(w)dw| \;=\; |f_r(r)dr|$), and we find it to be 
\begin{align}\label{eq:f_w}
 &\qquad f_w(w)dw \;=\; \frac{\Omega_d}{2}\left(\frac{\xi}{r_0}\right)^d\left|\ln \frac{w}{w_0}\right|^{d-1} \exp\left[-\left|\frac{\Omega_d}{2d}\frac{\xi}{r_0}\ln \frac{w}{w_0}\right|^d\right] \frac{dw}{w}
\end{align}
In $1d$ resistor networks, the average resistance of a system is the mean of the individual resistances, and the average conductance is the harmonic mean of the individual conductances. Following the $1d$ solution presented in \cite{Alexander:1981:RMP}, if the harmonic mean is finite, then the system will be diffusive with diffusion coefficient equal to the inverse of the harmonic mean:
\begin{align}
    D \;=\; \left(\overline{(w^{-1})}\right)^{-1}r_0^2 \;=\; \left[\int f(w) \frac{dw}{w}\right]^{-1} r_0^2
\end{align}
If $D>0$ then there is diffusion, else sub diffusion is implied. It is a known property of harmonic means, that because of their inverse dependence on values, the values closer to zero contribute most of the sum. Therefore, we are interested in the distribution of transition rates in the limit $w\rightarrow 0$. The $1d$ case of \eqref{eq:f_w} is:
\begin{align}\label{eq:1d_D}
f(w)dw &\;=\; \frac{\xi}{r_0} \left(\frac{w}{w_0}\right)^{\frac{\xi}{r_0} }\frac{dw}{w} \\
D &\;=\; \left[\int_0^\infty  \frac{\xi}{r_0} \left(\frac{w}{w_0}\right)^{\frac{\xi}{r_0} } \frac{dw}{w^2}\right]^{-1} r_0^2 \;=\; 
\begin{cases}
    \frac{\frac{\xi}{r_0} -1}{\frac{\xi}{r_0}}w_0^{\frac{\xi}{r_0}} r_0^2 &\qquad \frac{\xi}{r_0} > 1\\
    0  &\qquad \frac{2\xi}{r_0} < 1
\end{cases}
\end{align}

\begin{figure}

    \includegraphics[clip, width=0.49\hsize]{{{exp_1d_0.2_pn}}}   
    \includegraphics[clip, width=0.49\hsize]{{{exp_2d_0.2_pn}}}   \\
    \includegraphics[clip, width=0.49\hsize]{{{exp_1d_1.5_pn}}}   
    \includegraphics[clip, width=0.49\hsize]{{{exp_2d_1.5_pn}}}   \\
    \includegraphics[clip, width=0.49\hsize]{exp_1d_5_pn}         
    \includegraphics[clip, width=0.49\hsize]{exp_2d_5_pn}         
\caption{The exponential dependence model in $1d$ and $2d$. The sites' locations were randomly chosen with a uniform distribution. The transition rates between them are according to \eqref{eq:exp_dep}, with a different value of $\epsilon=\frac{\xi}{r_0}$ for each graph, as stated in their legend. For the $1d$ systems, where applicable, the expected eigenvalue density of diffusion modes was drawn, according to \eqref{eq:C_D} and \eqref{eq:1d_D}. For $2d$ systems, or for $1d$ systems with $D=0$, the diffusion density was drawn using Resistor Network calculation\eqref{eq:resnet}. }
\end{figure}

%%%%%%%%%%%%%
%%%%%ENDING THE DOCUMENT. WORK IN PROGRESS
%\bibliographystyle{plainnat}
\bibliography{jarondl}
\end{document}



{\bf Resistor network.--}
There is an analogy between the diffusion problem we have laid out, to electrical conductance. According to Fick's first law of diffusion, the mass flux is proportional to the gradient of the density:
\begin{align}
J = -D \nabla \phi 
\end{align}
{\bf Resistor network in $1d$.--} We can find the diffusion coefficient $D$ using resistor network calculation. The basic idea is to apply incoming current (+1 in suitable units) in one site, outgoing in another (-1), and then compute the site occupation probability difference between them. This will give us the conductivity $G$, which in $1d$ is the conductance $\sigma$ times the distance:
\begin{align}
  \sigma \;=\; \frac{G}{r}
\end{align} 
%%%%%%%%%%%%%%%%
{\bf Resistor network in $2d$.--} We do not know yet what is the exact criterion for diffusion in dimensions higher than $1d$. If we put in the $2d$ distribution, the harmonic mean does not diverge. Hence, if the naive generalization from $1d$ to $2d$ holds, we should observe diffusion. A different approach is to resolve the resistor network for a $2d$ system. Instead of solving the problem with both current connections, we can think of each of them seperately, and then superpose them. If a current goes from a single spot to infinity, the current should have radial symmetry. If so, the voltage at distance $r$ will be :
\begin{align}
  E \;=\; \frac{J}{\sigma} \;=\; \frac{I}{2\pi r\sigma} \\
  V \;=\; \int E dr \;=\; \frac{I}{2\pi\sigma}\ln\frac{r}{r_0}
\end{align}
The superposition of an outgoing current just doubles the result, and we obtain:
\begin{align}
    G \;=\; \frac{I}{V} \;=\; \frac{\pi\sigma }{\ln\frac{r}{r_0}}  \\
    \sigma \;=\; \frac{1}{\pi}G\ln\frac{r}{r_0}
\end{align}





