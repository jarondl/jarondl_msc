\chapter{Background}

 
\section{Transition matrices}

Our system is a $d$-dimensional network, and it is
described by its nodes and edges. The nodes represent 
either {\em sites} in real space or {\em levels} in energy space, 
and our particle (or particles, see \ref{sec:discreteness}) may
occupy them. The edges of the network represent transition rates 
between the nodes. The dynamics of the model are described by 
a rate equation:
%
\begin{align}
\frac{dp_n}{dt} \quad = \quad \sum_m w_{nm}p_m
\end{align}
%
Where the transition matrix 
%
\begin{align}
\mathbf{W} \quad =  \quad \left[ w_{nm}\right]_{N\times N}
\end{align}
%
Is the crux of this work.

Because of conservation of probability, we set the diagonal
elements to
%
\begin{align}
w_{nn}\quad = \quad -\sum_{m(\ne n)} w_{mn}
\end{align}
%


%%%%%%%%%%%%%%%%%%%%%%%%%%%%%%
\section{Definition of the model}


The proposal regards a spatial network in which each node
has a defined location $x_n$. The transition rates $w_{nm}$
are constructed by the expression
%
\beq
w_{nm} \quad = \quad w_0 \eexp{-\epsilon_{nm}}B(x_n-x_m)
\eeq
%
where $B(r)$ describes the dependence of the coupling on the
distance between sites, and $\epsilon$ represents activation
energies required for a transition.


The $B(r)$ matrix is a function of a euclidean distance matrix,
a class a matrices that is widely researched 
\cite{skipetrov_eigenvalue_2011, goetschy_non-hermitian_2011,mezard_spectra_1999, bogomolny_spectral_2003}
both in physics and in mathematics.


The disorder in these models arises from two origins, 
the random activation energy $\epsilon$, and the random distribution of the sites.


%%%%%%%%%%%%%%%%%%%%%%%%%%%%%%
\subsection{Applications of the model}


The models we define are abstract by nature, and
the spectrum of physical and statistical problems they
relate to is broad.
As physical examples, we can consider
phonon propagation in disordered solids 
\cite{nagel_normal-mode_1981,schirmacher_analogies_1992,amir_localization_2010,amir_mean-field_2008};
impurity conductance \cite{miller_impurity_1960,movaghar_theory_1981};
conductance of ballistic rings \cite{bandopadhyay_conductance_2006,*cohen_kubo_2007,*stotland_random-matrix_2010};
and energy absorption by trapped atoms \cite{stotland_semilinear_2009,*stotland_quantum_2010,*stotland_weak_2011}. 

%% Epidemic spreading, information spreading etc.

%%%%%%%%%%
\subsection{The degenerate random site hopping model} \label{sec:degenerate_random_hopping}

In this specific model, we set all of the activation energies ($\epsilon_{nm}$)
to be equal. We can choose $\epsilon=0$ without loss of generality. For $B(r)$, 
we have selected a simple exponential dependence with a parameter $\xi$.
Summing this up we write:
%
\begin{align}\label{eq:random_hopping_rates}
w_{nm} \quad = \quad w_0 B(x_n-x_m) \ = \ w_0 \eexp{-(x_n-x_m)/\xi}
\end{align}
%
Now we have defined the rates, but to characterize the system we also
have to define the location of the sites. In this model, $N$ sites are distributed
uniformly randomly on a periodic $d$-dimensional hypercube of volume $L^d$. The
local distribution of sites around any arbitrary point in this space is:
%
\begin{align}\label{eq:site_distribution}
\rho(r)dr \ =\ \frac{\Omega_d r^{d-1}}{r_0^d} dr
\end{align}
%
$\Omega_d$ and $r_0$ are defined in \autoref{sec:spacing}

%%%%%%%%%%%%
\subsection{The Mott hopping model}\label{sec:mott_hopping}

This model is an extension of the degenerate model to include
random values for $\epsilon$. $B(r)$ stays the same as in the degenerate model.
The rates are now:
%
\begin{align}\label{eq:mott_hopping_rates}
w_{nm} \quad = \quad \ w_0 \eexp{-\epsilon_{nm}}\eexp{-(x_n-x_m)/\xi}
\end{align}
%
And the combined distribution of $\epsilon$ and $r$ is now:
%
\begin{align}\label{eq:mott_distribution}
\rho(r)f(\epsilon)drd\epsilon \ &=\ \frac{\Omega_d r^{d-1}}{r_0^d} f(\epsilon) dr d\epsilon \\
f(\epsilon)\ &= 
  \begin{cases} 
    1 &\textrm{if   } \epsilon \in [0,\sigma] \\
    0 &\textrm{otherwise}
  \end{cases}
\end{align}
%

Not so relevant. 2d lattice with random hopping, but only 2 options. It has a calculation of the Diffusion tensor.
%%%%%%%%%%%%%
\subsection{The banded $1d$ lattice model}

In this model, we have a $1d$ lattice (i.e.\ equal spacing between the sites), 
with the rates :
%
\begin{align}
w_{nm} \quad = \quad \eexp{-\epsilon_{nm}}w_0 B(x_n-x_m)
\end{align}
%
The profile $B(r)$ allows connections to finite number of neighbors,
meaning that the resulting matrix is banded. This model might also
be called quasi-$1d$ because transitions could occur over more
than one path, removing the "bottle-neck" effect of $1d$ systems.


%%%%%%%%%%%%%%
\section{Sparsity}\label{sec:sparsity}

In general, a sparse network has few edges per node. Sparse networks 
are often constructed by forming a dilute network, where each site
has a finite amount of connected edges 
\cite{rodgers_density_1988,biroli_single_1999,fortin_asymptotic_2005,metz_localization_2010}.
However, we consider as sparse even networks with many edges per site
if a only \emph{small} portion of the edges dominate, and a \emph{large}
portion is almost zero \cite{cohen_energy_2012,stotland_semilinear_2009}. In our system, the bonds span
over several orders of magnitude, with the \emph{sparsity} 
determined by the value of $\xi/r_0$ and by the width of the distribution
of $\epsilon$.

%%%%%%%%%%%%%%%%%%%
%% Percolation?



%%%%%%%%%%%%%%%%%
\section{Transport and diffusion}

One may characterize the transport properties of materials 
by the spreading $S(t)$ , which is the second moment of 
the particle location, and the survival probability $\mathcal{P}(t)$, i.e.:
%
\begin{align}
S(t) &= \sum_n (r_n)^2 p_n(t) \\
\mathcal{P}(t) &= p_0(t)
\end{align}
%
Where $r_n$ is the distance of site $n$ from the origin,
and $p_n(t)$ is the probability of occupancy in site $n$, and
the initial distribution was $p_n(0) = \delta_{n,0}$.


In a diffusive system, the coarse grained distribution is an evolving Gaussian
%
\begin{align}
p_n(t) = \prod_{i=1}^d \frac{1}{\sqrt{2\pi\ 2Dt}} \exp\left[-\frac{1}{2}\frac{x_i^2}{2Dt}\right]
\end{align}
%
Meaning that the spreading and survival probability are:
%
\begin{align}
S(t) \ &=\ (2d)\ D\ t \\
\mathcal{P}(t)\ &=\ \frac{1}{(4\pi D t)^{d/2}}
\end{align}
%


%%%%%%%%%%%%%%%%%%%%%%
\section{Resistor network calculation}

Apart from the spreading and survival, the transport may
also be characterized by the conductivity of the network
\cite{halperin_remarks_1989,miller_impurity_1960,ambegaokar_hopping_1971,pollak_percolation_1972,
Aharony_effective_1991,stotland_random-matrix_2010,stotland_semilinear_2009}.
The diffusion coefficient $D$ is formally equal (up to unit
changes) to the conductivity of the network.


The conductivity can be extracted from the 
numerically solved conductance, as explained in \autoref{sec:resnet}

%%%%%%%%%%%%%%%%%%%%%%%
\section{Extracting the transport from spectral properties}\label{sec:spectrum}

Let us define the spectral function $\mathcal{N}(\lambda)$ as
the integral over the local density of states $g(\lambda)$:
%
\begin{align}
\mathcal{N}(\lambda)\ =\ \int_0^\lambda g(\lambda')d\lambda'
\end{align}
%
It is well known that the survival probability is related to the spectrum through the relation
%
\begin{align}
\mathcal{P}(t)\ =\ \frac{1}{N}\sum_\lambda \eexp{\lambda t} \ \equiv \ \int_0^\infty g(\lambda)d\lambda  \eexp{\lambda t}
\end{align}
%
Which gives more emphasis to the \emph{lower} end of the spectrum.
In a diffusive system, the eigenvalues of the diffusion are $\lambda=Dq^2$, where the possible values of the momentum $q$ are determined by the periodic boundary conditions as $q=(2\pi/L)\vec{n}$. It follows that the spectral function is:
 %
\begin{align}
\mathcal{N}(\lambda)\ =\ \left(\frac{L}{2\pi}\right)^d \frac{\Omega_d}{d} \left[\frac{\lambda}{D}\right]^{d/2}
\end{align}
%
Therefore we can use matrix diagonalization to find the lowest
eigenvalues, and derive the diffusion coefficient.

%%%%%%%%%%%%%%%%%%%%%%%

%\section{Variable range hopping}


%\section{Anderson localization}


%%%%%%%%%
%\section{Analogy to an electrical circuit}

%We may think of a "ladder" of capacitors and resistors as portrayed in the diagram.



%%%%%%%%%
%\section{Analogy to a masses and springs mesh} 

%The same network may describe equal masses (which 
%we can simply define as $1$) distributed in space,
%connected by springs. The locations of the masses will
%obey a second degree differential equation:


%%%%%%%%%
%\section{Spreading of wavepackets in the quantum case}





