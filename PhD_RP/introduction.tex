\chapter{Introduction}


This proposal is about simple dynamical networks, with dynamics
that are described either by a stochastic rate equation
%
\begin{align}
  \frac{dp_n}{dt} \quad = \quad \sum_m w_{nm}p_m
\end{align}
%
or by the Schr\"{o}dinger equation:
%
\begin{align}
  \frac{d\psi_n}{dt} \quad = \quad \sum_m \mathcal{H}_{nm}\psi_m
\end{align}
%

Regardless of the governing equation, the focus is
on disordered sparse networks, with wide distribution of energies. 
Two related aspects of these networks will be studied: their energy spectrum,
and their spreading properties. More specifically, we wish to determine
whether the networks are diffusive ($S(t)\sim t$), or sub-diffusive, and
if they are indeed diffusive, what is the diffusion coefficient. 




To simplify the discussion, we concentrate on $d$-dimensional spatial networks,
in which each node has a defined location $x_n$, and each edge has an associated 
energy barrier $\epsilon_{nm}$. The transition rates are then defined to be:
%
\begin{align}
w_{nm} \quad = \quad w_0 \eexp{-\epsilon_{nm}}B(x_n-x_m)
\end{align}
%
Where $B(r)$ describes the dependence of the coupling on the
distance between sites.


The model we start with has its sites distributed randomly and uniformly throughout
a $d$-dimensional hypercube, with $B(r) = \exp(-r/\xi)$. The disorder is defined by the location
of the sites, and its effect on the connectivity of the network can be manipulated through the 
parameter $\xi$. 



%%%%%%%%%%%%%%%%%%%%%%%%%%%%%%%%%%%%
\section{TODO matrix categories}\label{sec:matrix_categories}

% Three categories:
% wnm = wmn  - no talk about steady state - equilibrium transient diffusion
%%            resnet
% wnm propto wmn - detailed balance - stochastic field
%           integral over it must be zero  boltzmann equil
%           should be added to the background
% wnm \ne wmn - assymetry - frustrated NESS

\bigskip

The organization of this proposal is in parts.
\autoref{part1} includes this introduction,
the objectives of the proposal and further background. 
In \autoref{part2} we present the work accomplished so far.
In \autoref{part3} lies the appendix, including some derivations,
and an article currently under review at PRE, available in the arXiv \cite{de_leeuw_diffusion_2012}.
