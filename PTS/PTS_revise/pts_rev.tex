%%%%%%%%%%%%%%%%%%%%%%%%%%%%%%%%%%%%%%%%%%%%%%%%%%%%%%%%%%%%%%%
%%%  pts
%%%%%%%%%%%%%%%%%%%%%%%%%%%%%%%%%%%%%%%%%%%%%%%%%%%%%%%%%%%%%%%
\documentclass[aps,prb,floats,floatfix,twocolumn]{revtex4}

% special 
\usepackage{ifthen}
\usepackage{ifpdf}
\usepackage{color}

\ifpdf
\usepackage{graphicx}
\usepackage{epstopdf}
\else
\usepackage{graphicx}
\usepackage{epsfig}
\fi

% fonts
\usepackage{latexsym}
\usepackage{amsmath}
\usepackage{amssymb}
\usepackage{bm}
\usepackage{wasysym}

%\usepackage{hyperref}

%%%%%%%%%%%%%%%%%%%%%%%%%%%%%%%%%%%%%%%%%%%%%%%%%%%%%%%%%%%%%%%%

% NEW 
\newcommand{\abs}[1]{\left|#1\right|}
\newcommand{\Prob}{\mbox{Prob}\,}
\newcommand{\erf}{\mbox{erf}\,}
\newcommand{\df}{{\rm d}}

% math symbols I
\newcommand{\sinc}{\mbox{sinc}}
\newcommand{\const}{\mbox{const}}
\newcommand{\trc}{\mbox{trace}}
\newcommand{\intt}{\int\!\!\!\!\int }
\newcommand{\ointt}{\int\!\!\!\!\int\!\!\!\!\!\circ\ }
\newcommand{\ar}{\mathsf r}
\newcommand{\im}{\mbox{Im}}
\newcommand{\re}{\mbox{Re}}

% math symbols II
\newcommand{\eexp}{\mbox{e}^}
\newcommand{\bra}{\left\langle}
\newcommand{\ket}{\right\rangle}

% Mass symbol
\newcommand{\mass}{\mathsf{m}} 
\newcommand{\Mass}{\mathsf{M}} 

% more math commands
\newcommand{\tbox}[1]{\mbox{\tiny #1}}
\newcommand{\bmsf}[1]{\bm{\mathsf{#1}}} 
%\newcommand{\amatrix}[1]{\matrix{#1}} 
\newcommand{\amatrix}[1]{\begin{matrix} #1 \end{matrix}} 
\newcommand{\pd}[2]{\frac{\partial #1}{\partial #2}}

% equations
\newcommand{\mylabel}[1]{\label{#1}} 
%\newcommand{\mylabel}[1]{\textcolor{blue}{[#1]}\label{#1}} 
\newcommand{\beq}{\begin{eqnarray}}
\newcommand{\eeq}{\end{eqnarray}} 
\newcommand{\be}[1]{\begin{eqnarray}\ifthenelse{#1=-1}
{\nonumber}{\ifthenelse{#1=0}{}{\mylabel{e#1}}}}
\newcommand{\ee}{\end{eqnarray}} 

% arrangement
\newcommand{\drawline}{\begin{picture}(500,1)\line(1,0){500}\end{picture}}
\newcommand{\bitem}{$\bullet$ \ \ \ }
\newcommand{\Cn}[1]{\begin{center} #1 \end{center}}
\newcommand{\mpg}[2][1.0\hsize]{\begin{minipage}[b]{#1}{#2}\end{minipage}}
\newcommand{\mpgt}[2][1.0\hsize]{\begin{minipage}[t]{#1}{#2}\end{minipage}}
\newcommand{\putgraph}[2][width=0.30\hsize]{\includegraphics[#1]{#2}}

% more
%\newcommand{\Eq}[1]{Eq.\!\!~(\ref{#1})}
%\newcommand{\Fig}[1]{Fig.\!\!~\ref{#1}}  
\newcommand{\Eq}[1]{\textcolor{blue}{Eq.\!\!~(\ref{#1})}} 
\newcommand{\Fig}[1]{\textcolor{blue}{Fig.}\!\!~\ref{#1}} 
\newcommand{\hide}[1]{}
%\newcommand{\hide}[1]{\textcolor{red}{[hidden text]}} %{}
\newcommand{\rmrk}[1]{\textcolor{red}{#1}}
%\newcommand{\Rmrk}[1]{\textcolor{blue}{\LARGE\bf #1}}

%\renewcommand{\includegraphics}[2][]{\ \\ \ FIGURE: \ \\ \ }
\renewcommand{\cite}[1]{\textcolor{blue}{[\onlinecite{#1}}]} %{[\onlinecite{#1}]} 


%%%%%%%%%%%%%%%%%%%%%%%%%%%%%%%%%%%%%%%%%%%%%%%%%%%%%%%%%%%%%
%%%%%%%%%%%%%%%%%%%%%%%%%%%%%%%%%%%%%%%%%%%%%%%%%%%%%%%%%%%%%
\begin{document}


\appendix
\section{The ERH calculation for the degenerate hopping model}

We wish to apply the ERH expression of \Eq{31} to the rates of the degenerate model.

The rates are:
%
\beq
w_{nm} \ \ = \ \  w_0 \ \eexp{-\epsilon_{nm}} \ \eexp{-r/\xi} 
\eeq
%
And the local density of sites is:
%
\beq
\rho(r,\epsilon) \ \ =\ \ \frac{\Omega_d r^{d-1}}{r_0^d} \ \delta (\epsilon)
\eeq
%
The ERH integral in \Eq{31} has two regimes, one in which $w>w_c \ \& \ r<r_c$, and the other in which $w>w_c\ \& \ r>r_c$.
The first integral is a trivial power-law integral, and the other is by defintion the upper incomplete gamma function\cite{dlmf_gamma}.
%
\beq
D_{\tbox{ERH}} = \frac{w_0\Omega_d}{2d}\int_0^{r_c} \eexp{-r_c / \xi} \frac{r^{d+1}}{r_0^d} dr 
               + \frac{w_0\Omega_d}{2d}\int_{r_c}^\infty \eexp{-r / \xi} \frac{r^{d+1}}{r_0^d} dr \\
\ \ \ = \frac{w_0\Omega_d}{2d} \eexp{-r_c / \xi} \frac{r_c^{d+2}}{d+2} \frac{1}{r_0^d} 
\ \ \  +\frac{w_0\Omega_d}{2d} \frac{\xi^{d+2}}{r_0^d} \Gamma\left(d+2,\frac{r_c}{\xi}\right)\\
\ \ \ = \frac{w_0\Omega_d \xi^{d+2}}{2d(d+2)r_0^d}\Gamma\left(d+3,\frac{r_c}{\xi}\right)
\eeq
%
Some notational changes for convenience lead us directly to \Eq{40} 

%note!
%below eq (40) there is a reference to  $R_l$

%%%%%%%%%%%%%%%%%%%%%%%%%%%%%%%%%%%%%%%%%%%%%
%%%%%%%%%%%%%%%%%%%%%%%%%%%%%%%%%%%%%%%%%%%%%
\begin{thebibliography}{99}
\bibitem{dlmf_gamma}
Digital Library of Mathematical Functions. 2012-03-23. National Institute of Standards and Technology  http://dlmf.nist.gov/8



\end{thebibliography}

\end{document}


