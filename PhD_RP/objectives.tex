\chapter{Objectives}

%TODO: orientation.

The first two objectives have been the target of my MSc, and are covered
in an article published at Physical Review E \cite{de_leeuw_diffusion_2012},
and reprinted here in \autoref{sec:papers}. 
The other objectives will be pursued during my PhD research.


%%%%%%%%%
\section{Diffusion or Subdiffusion in $2d$}

For the stochastic random network, the question of diffusion in most $1d$
systems was analytically solved in \cite{alexander_excitation_1981}. 
The $2d$ case is, as far as we know, not yet analytically solved, and is much less clear. 
In \cite{amir_localization_2010} it is claimed that in low density systems subdiffusion of
order $\sim log^d$ should occur. We wish to examine these networks through their spectral
properties to find out if this is indeed the case. Our focus will be on low densities,
and on checking whether there exists a transition between low density and high density behavior.

Another aspect we wish to investigate is the influence of geometrical and 
topological features on the physics of the system, beyond the direct 
effect of changing the matrix elements statistics.

%%%%%%%%%
\section{Banded sparse matrices}


Between the $1d$ nearest neighbor network and the $2d$ network,
lie the quasi $1d$ network, which allows transitions to further neighbors.
The conductance of quasi-$1d$ banded sparse matrices was studied
numerically in \cite{stotland_random-matrix_2010}.
There, they generalized the \emph{Variable-Range-Hopping} scheme to treat 
matrices in the sparse regime, where 
$(\text{sparsity}\cdot \text{bandwidth}) \ll 1$. 
However, in this work it is not clear what are the limits on either
sparsity or bandwidth, and in particular where is the cross between the 
validity regime of VRH and that of SLRT. We will try to understand this
issue analytically.



% TODO A specific version of these matrices were also studied in \cite{janssen_correlated_2000}.

%%%%%%%%%%%
\section{Models with relaxation}

%% same potentials - is not relevant.
%% do I mean Steady State?
% assymetric transitions
% Boltzman - detailed balance
%% remove assymetric VRH
% models with relaxations

% the point is that I want to deal with assymetry.
%% "generaly the transition matrices are assymetric,
%% but up to now we dealt with symmetric only."
%% why combine on-site energy with assymetry.

% Three categories:
% wnm = wmn  - no talk about steady state - equilibrium transient diffusion
%%            resnet
% wnm propto wmn - detailed balance - stochastic field
%           integral over it must be zero  boltzmann equil
%           should be added to the background
% wnm \ne wmn - assymetry - frustrated NESS

Up to now we have dealt with symmetric matrices only, but as presented in 
the introduction \autoref{sec:matrix_categories}, in general the transition 
rates may be assymetric. We wish to extend the methods used so far to
develop a better estimation method for the transport in these assymetrical matrices.


% stotland Errata variable activation energy.. 
%%%%%%%%%
%\section{stotland's errata model}



%%%%%%%%%%%%%%%%%%%%  HEAT must appear!
%%%%%%%%%%%%%%%%%%%% I'm in charge of heat
%% bullets - where does this model appear - balls and springs (leads to heat) - capacitors
%%             - conductance hopping

\section{Heat transport}

\begin{comment}
Heat conduction by phonons is also affected by the localization properties
of the model. If disorder scatters normal modes and induces diffusive 
energy transport, followed by normal heat conduction, then in accordance to Fourier's law, 
the heat current $J$ depends inversely on the system size: $J\sim N^{-1}$. 
However, in recent studies \ref{who} it was shown that $J\sim N^{-\alpha}$ 
with $\alpha\ne 1$ is sometimes the case for disordered $1d$ harmonic chains.
We wish to investigate the validity of Fourier's law for quasi-$1d$ and 
higher dimensional disordered systems.
\end{comment}

As presented in the introduction, heat conduction by phonons has some
similarities with other transport properties. We wish to relate as much
as possible the various definitions of conductance coefficients, in 
particular the relation between the spectral analysis,
the resistor network computation based on Kirchoff's equation
and the covariance matrix formalism Lyapunov equation. Our initial focus
will be on banded sparse matrices, since these provide methods to distinguish
between percolative effects and disorder.


%%%%%%%%%
\section{The spectral properties of Sinai diffusion in $1d$}

In a symmetric $1d$ random rate system, there is regular diffusion 
if the harmonic average is bounded, i.e.\ if
%
\begin{align}
 \sum_n \frac{1}{w_n} < \infty 
\end{align}
%

On the other hand, for an asymmetric system, an activation energy 
builds up, and a Sinai diffusion behavior is observed.
The conductance is exponentially small in the length, and the spreading 
is anomalous. The difference between those two models is clearly highly 
significant. We wish to investigate the difference between the models,
and the manifestation of this difference in the spectral properties of the model.

%%%%%%%%%
\section{Sinai diffusion in $2d$}

We shall inspect the spectral properties, and spreading behavior of a
$2d$ asymmetric system. The $2d$ case is much different than the $1d$ 
case, because particles can circumvent barriers. Interestingly,
a nearest neighbor lattice can be constructed to present the same 
long time spreading as the $1d$ lattice \cite{blumberg_selinger_diffusion_1989}.
The question is whether having nearest neighbor links only 
is a necessary condition for this behavior,
and how does this relate to the spectral properties of the $2d$ system.



%%%%%%%%%%
\section{Quantum spreading}

Up to now, we have concentrated on Markovian networks, dealing with the probabilities without phases.
We will try to generalize the work to solve Hamiltonian matrices of the same genre.
With a coherent Hamiltonian, 
we expect suppression of the diffusion due to Anderson localization.
However there may be transient diffusion, which would persist 
for infinite time in the presence of dephasing.

We intend to examine the applicability of the resistor network picture for 
the calculation of the diffusion coefficient in these quantum networks, beginning with
a simple model described by a real symmetric Hamiltonian:
%
\begin{align}
  \mathcal{H}_{nm} = \textrm{random}[\pm] \eexp{-\textrm{random}[x]} \qquad 0<|n-m|\le b
\end{align}
%
Where the parameter $x$ is distributed uniformly $x\in [0,\sigma]$. 

%%%%%%%%%% many body must appear after the quantum case

%%%%%%%%%
\section{The charge carrier discreteness}\label{sec:discreteness}

Our work so far considered probability rate equations and single-particle Hamiltonians, which are valid 
for a single particle in the network.
Having many non interacting particles is equivalent to having many 
realizations of a single particle, leaving the picture intact. 
However, if we have multiple \emph{interacting} particles in the network, 
additional factors may contribute. For example, a single occupancy rule 
(i.e.\ each site may have up to one particle) may cause Fermi blocking. 
Another issue is that the single particle behavior might differ from
the bulk behavior, e.g.\ it could be sub-diffusive 
while the bulk spreading is diffusive \cite{richards_theory_1977,hung_diffusion_2012}.




% Kottos. Quantum version. Localization etc. 
