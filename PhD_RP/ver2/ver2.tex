%%%%%%%%%%%%%%%%%%%%%%%%%%%%%%%%%%%%%%%%%%%%%%%%%%%%%%%%%%%%%%%
%%%  pta notes
%%%%%%%%%%%%%%%%%%%%%%%%%%%%%%%%%%%%%%%%%%%%%%%%%%%%%%%%%%%%%%%

\documentclass[onecolumn,fleqn,longbibliography]{revtex4}

% fonts
\usepackage{latexsym}
\usepackage{amsmath} 
\usepackage{amssymb} 
\usepackage{bm}
\usepackage{wasysym}

\usepackage{graphicx}


% extra by jarondl

\usepackage{array}
\usepackage{float}%unfloats
%\usepackage{multicol}
\usepackage[caption=false]{subfig} %subcaption is not compat with revtex
\usepackage[pdftitle={PTA},bookmarks]{hyperref}

%%%%%%%%%%%%%%%%%%%%%%%%%%%%%%%%%%%%%%%%%%%%%%%%%%%%%%%%%%%%%%%%

% NEW 
\newcommand{\abs}[1]{\left|#1\right|}
\newcommand{\varphiJ}{\bm{\varphi}}
\newcommand{\thetaJ}{\bm{\theta}}
%\renewcommand{\includegraphics}[2][0]{FIGURE}
\newcommand{\rmrk}[1]{\textcolor{red}{#1}}
\newcommand{\Eq}[1]{\textcolor{blue}{Eq.\!\!~(\ref{#1})}} 
\newcommand{\Fig}[1]{\textcolor{blue}{Fig.}\!\!~\ref{#1}}

% math symbols I
\newcommand{\sinc}{\mbox{sinc}}
\newcommand{\const}{\mbox{const}}
\newcommand{\trc}{\mbox{trace}}
\newcommand{\intt}{\int\!\!\!\!\int }
\newcommand{\ointt}{\int\!\!\!\!\int\!\!\!\!\!\circ\ }
\newcommand{\ar}{\mathsf r}
\newcommand{\im}{\mbox{Im}}
\newcommand{\re}{\mbox{Re}}

% math symbols II
\newcommand{\eexp}{\mbox{e}^}
\newcommand{\bra}{\left\langle}
\newcommand{\ket}{\right\rangle}

% Mass symbol
\newcommand{\mass}{\mathsf{m}} 
\newcommand{\rdisc}{\epsilon} 

% more math commands
\newcommand{\tbox}[1]{\mbox{\tiny #1}}
\newcommand{\bmsf}[1]{\bm{\mathsf{#1}}} 
\newcommand{\amatrix}[1]{\begin{matrix} #1 \end{matrix}} 
\newcommand{\pd}[2]{\frac{\partial #1}{\partial #2}}

% equations
\newcommand{\mylabel}[1]{\label{#1}} 
\newcommand{\beq}{\begin{eqnarray}}
\newcommand{\eeq}{\end{eqnarray}} 
\newcommand{\be}[1]{\begin{eqnarray}\ifthenelse{#1=-1}{\nonumber}{\ifthenelse{#1=0}{}{\mylabel{e#1}}}}
\newcommand{\ee}{\end{eqnarray}} 

% arrangement
\newcommand{\hide}[1]{}
\newcommand{\drawline}{\begin{picture}(500,1)\line(1,0){500}\end{picture}}
\newcommand{\bitem}{$\bullet$ \ \ \ }
\newcommand{\Cn}[1]{\begin{center} #1 \end{center}}
\newcommand{\mpg}[2][1.0\hsize]{\begin{minipage}[b]{#1}{#2}\end{minipage}}
\newcommand{\mpgt}[2][1.0\hsize]{\begin{minipage}[t]{#1}{#2}\end{minipage}}


%footnotemark:
\renewcommand*{\thefootnote}{\fnsymbol{footnote}}
%%%%%%%%%%%%%%%%%%%%%%%%%%%%%%%%%%%%%%%%%%%%%%%%%%%%%%%%%%%%%%%%%%%%%%%%%%%
% Sections
\newcommand{\sect}[1]
{
\addtocounter{section}{1} 
\setcounter{subsection}{0}
\ \\ 
\pdfbookmark[2]{\thesection. \ #1}{sect.\thesection}
{\Large\bf $=\!=\!=\!=\!=\!=\;$ [\thesection] \ #1}  
\nopagebreak
}

% subections
\newcommand{\subsect}[1]
{
\addtocounter{subsection}{1} 
\ \\ 
\pdfbookmark[2]{\ \ \ \ \thesection.\thesubsection. \ #1}{subsect.\thesection.\thesubsection}
{\bf $=\!=\!=\!=\!=\!=\;$ [\thesection.\thesubsection] \ #1}  
\nopagebreak
}
%%%%%%%%%%%%%%%%%%%%%%%%%%%%%%%%%%%%%%%%%%%%%%%%%%%%%%%%%%%%%%%%%%%%%%%%
%%%%%%%%%%%%%%%%%%%%%%%%%%%%%%%%%%%%%%%%%%%%%%%%%%%%%%%%%%%%%

\graphicspath{{figures/}}
\begin{document}

\title{PhD resubmission}

\author{YdL}

\maketitle

%%%%%%%%%%%%%%%%%%%%%%%%%%%%%%%%%%%%%%%%%%%%%%%%%%%%%%%%%%%%%%%%%%%%%%%%
%%%%%%%%%%%%%%%%%%%%%%%%%%%%%%%%%%%%%%%%%%%%%%%%%%%%%%%%%%%%%%%%%%%%%%%%

\sect{Targets}

\begin{itemize}
\item
Better summarize the background in heat transport, according to wiki subjects.
\item
We have two matrices. One 
\end{itemize}

\sect{Ohm's law}

According to Datta, the "new" Ohm law (in $2d$ with width $W$) is:
\begin{align*}
G = \frac{\sigma W}{L+\lambda}
\end{align*}
Where $\lambda$ is the mean free path. That expression should 
work for 

\sect{Matrices as applied for conduction}

This is the current question.

As we wrote in the proposal, we have matrices, they can have many 
differing physical meanings. Let's say I can diagonalize the matrix.
How do I find $G$, $\kappa$ and $D$?

It seems that for $G$ we need the spectrum around $\varepsilon_F$,
for $D$ we need the spectrum around $0$, and for $\kappa$ we need 
the entire spectrum.

% Leibowitz, maybe their complexity arises
% from different masses.

%% compare Landauer and widen miller to the langevin result.
%% why sometimes it is ohmic and sometimes not. Is our model even ohmic?


%For an ordered potential with no disorder, we have no Ohm's law, 
% just transmission 1
\end{document}
