\chapter{Completed Work}

% Use drawings and slides from the theory lunch (actual and prelim talk)

% Also anything I have in the notes which is worthy.
% For example distance statisics. If I did not conrtibute, should be in the Appendix.
% Half cooked ideas..


%%%%%%%%%%%%%%%%%%%%%%%%%
\section{The ERH procedure}

The diffusion coefficient is defined by the dependence
of the variance in time:
%
\begin{align}
\textrm{Var}(n) \quad =\quad 2d\ D\ t
\end{align}
%
Within the transient diffusion stage,
assuming we start at a specific node $n$, the variance after time
$t$ will be
%
\begin{align}
\textrm{Var}(n) = \sum_m w_{nm} t (x_m-x_n)^2
\end{align}
Therefore, the diffusion coefficient for evolution 
starting at node $n$ is
%
\begin{align}
D_n \quad=\quad \frac{1}{2d}\sum_m (x_m-x_n)^2 w_{nm}
\end{align}
%
Averaging over the starting node, and converting the sum to an integral,
we have
%
\begin{align}
D_{\tbox{linear}}  \ \ = \ \ \frac{1}{2d}\iint w(r,\epsilon) \ r^2  \ \rho(r,\epsilon) \ d\epsilon dr 
\end{align}
%

This expression describes the spreading in absence of disorder correctly
for arbitrary long time. However, in a sparse disordered system such as ours,
the possibility of transport is a bit less trivial. Therefore, we suggest
a method to approximate the diffusion, which takes percolation into account.


The basic idea behind \emph{ERH} is that in the linear
expression, nearby sites 
with $r\ll 1$ (and therefore $w \gg 1$ ) are over represented in
the diffusion coefficient calculation. While the transition to
these sites is indeed high, the distance covered is not enough
to form a percolating cluster. Therefore, we use a threshold based
on percolation theory and flat-down the rates higher then this threshold.

The rate threshold $w_c$ is determined using the following expression
%
\begin{align}
\iint_{w(r,\epsilon)>w_c} \rho(r,\epsilon)drd\epsilon \ \ = \ \ n_c
\end{align}
%
Where $n_c$ is the number of connected sites necessary to form a percolation
cluster (see \rmrk{TO DO :ref to section explaining this better}).

%%%%%%%%%%%%%%%%%%%%%%%%%
\section{The $d=1$ lattice model}

%%%%%%%%%%%%%%%%%%%
\subsection{Exact and Numerical results}


%%%%%%%%%%%%%%%%%%%%%%%%%
\section{The $d=2$ lattice model}

%%%%%%%%%%%%%%%%%%%
\subsection{Numerical results}

%%%%%%%%%%%%%%%%%%%
\subsection{ERH calculation}



